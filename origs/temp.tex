
\title{Introduction}

Carmen Pérez-Vidal\textsuperscript{1}, Sonia López-Seranno\textsuperscript{1,2}, Jennifer Ament\textsuperscript{1,3}, Dakota Thomas-Wilhelm\textsuperscript{4,5}

\textit{\textsuperscript{1}}\textit{Universitat} \textit{Pompeu} \textit{Fabra,} \textit{\textsuperscript{2}}\textit{Universidad} \textit{de} \textit{Murcia,} \textit{\textsuperscript{3}}\textit{Universitat} \textit{Internacional} \textit{de} \textit{Catalunya,} \textit{\textsuperscript{4}}\textit{University} \textit{of} \textit{Iowa,\textsuperscript{} } \textit{\textsuperscript{5}}\textit{Universitat} \textit{Autònoma} \textit{de} \textit{Barcelona}

This volume within the EuroSLA Studies Series has been motivated by two fundamental reasons. Firstly, the assumption that applied linguistics research should first and foremost deal with topics of great social relevance, and, secondly, that it should also deal with topics of scientific relevance. Both ideas have led us to choose the theme ‘contexts of language acquisition’ as the topic around which the monograph would be constructed.\footnote{This work was supported by the AGENCIA UNIVERSITARIA DE RECERCA (AGAUR), in Catalonia, [2014 SGR 1568]; by the Ministry of Economy and Competitiveness [FFI2013-48640-C2-1-P], and by a EUROSLA workshop grant (2017).   The monograph follows the EUROSLA workshop on the same theme celebrated at the Universitat Pompeu Fabra in Barcelona, Spain, 23-24 May, 2017.}

The aim of this introduction is to set the scene and present the three contexts on focus in the monograph and justify this choice of topic within second language acquisition (SLA) research, the perspective taken in this volume. Starting with the latter, in the past two decades the examination of the effects of different contexts of acquisition has attracted the attention of researchers, based on the idea that “the study of SLA within and across various contexts of learning forces a broadening of our perspective of the different variables that affect and impede acquisition in general” (\citealt{CollentineFreed2004b}: 157). The authors continue, “however, focusing on traditional metrics of acquisition such as grammatical development might not capture important gains by learners whose learning is not limited to the formal classroom (ibid: 158)”. With reference to the social relevance of the topic, European multilingual policies in the past decades have been geared towards the objective of educating our young generations in order to meet the challenge of multilingualism (\citealt{Coleman2015}; Pérez-\citealt{Vidal2015a}), ultimately as an effect of “globalization and the push for internationalization [on campuses] across the globe” \citep[1]{Jackson2013}. Indeed, the majority of European member states have embraced the recommendations made by the Council of Europe, encapsulated in the well-known 1+2 formula, according to which European citizens should have democratic access to proficiency in their own language(s) plus two other languages. In order to reach such a goal, a couple of decades ago the Council of Europe put forward a series of key recommendations to member states: i) an earlier start in foreign language learning; ii) mobility (the European Action Scheme for the Mobility of University Students, ERASMUS, exchange programme was launched in 1987, and since then more than three million students have benefitted from it); and iii) bilingual education, whereby content subjects should be taught through a foreign language (Commission of the European \citealt{Communities1995}, 2013). The latter recommendation has given rise to a number of immersion programmes at primary, secondary and tertiary levels of education, in parallel to the existing elite international schools (see the Eurobarometer figures and \citealt{WächterMaiwörm2014}, respectively). Such programmes are mostly taught through English, but also through French, German, Catalan, and other languages. Whether such learning contexts, which we have called ‘international classrooms’ and include classrooms at home and abroad (Pérez-Vidal, \citealt{LorenzoTrenchs2017}), are de facto conducive to language acquisition is a matter which indeed needs to be investigated. 

Against such a backdrop, this research monograph deals with the effects of different learning contexts mainly on adult, but also on adolescent learners’ language acquisition. More specifically, it aims at comparing the effects of three learning contexts by examining how they change language learners’ linguistic performance, and non-linguistic attributes, such as motivation, sense of identity and affective factors, as has been suggested not only by \citet{CollentineFreed2004b} mentioned above, but also by a number of other authors (to name but a few, \citealt{Pellegrino2005}; \citealt{Dewaele2007}; \citealt{Hernández2010}; \citealt{LasagabasterDoiz2014}; Taguchi, \citealt{XiaoLi2017}). 

More specifically, the three contexts brought together in the monograph include i) a conventional instructed second language acquisition (ISLA) context, in which learners receive formal instruction (FI) in English as a Foreign Language (EFL); ii) a study abroad (SA) context, which learners experience during mobility programmes, with the target language no longer being a foreign but a second language, learnt in a naturalistic context; iii) the immersion classroom, also known as an integrated content and language (ICL) setting, in which learners are taught content subjects through the medium of the target language - more often than not English, hence the term  English-Medium Instruction (EMI), and possibly English as a Lingua Franca (ELF) (\citealt{Björkman2013}; \citealt{House2013}). One last point needs to be made, concerning the issue of internationalisation, as is clearly stated in the title of the monograph:at any rate, the three contexts of acquisition on focus in this volume represent language/culture learning settings in which an \textit{international} \textit{stance} may be promoted in learners, as described below, in some cases also including the internationalization of the curriculum \citep{Leask2015}. 

In the SLA tradition in which the different chapters contained in the volume are framed, the comparison across contexts has been established under the assumption that contexts vary in the “type of input received by the learner (implicit vs. explicit), the type of interaction required of the learner (meaning-focus vs. form-focused)” (\citealt{LeonardShea2015}: 185), and, most importantly, the type of exposure to the target language, with variations in the amount of “input, output and interaction opportunities available to them” (Pérez-\citealt{Vidal2014}: 23). As the focus is on three different learning contexts - SA, EMI, and FI - we suggest that they can be understood as situated on a continuum in which the most “interaction-based”, with more favorable quantity and quality of input, would occur during a SA period. Second in order would be a semi-immersion context, as might take place EMI programmes, and the most “classroom-based” being FI in ISLA. Similarly, it is also along such a continuum, that these contexts make possible for learners to develop an attribute which \citet{UshiodaDörnyei2012} refer to as an \textit{international} \textit{stance}. That is to say, learners have the opportunity to incorporate a new view of the world that integrates languages and cultures other than their own, often through the use of English as a lingua franca as a means of communication.

Turning to the cognitive mechanisms made possible in different linguistic environments or learning contexts, these have ultimately also been claimed to be different. De\citet[213]{Keyser2007} draws on skill acquisition theory, which distinguishes three stages -  declarative knowledge, procedural knowledge and automatization -  to suggest that, “a stay abroad should be most conducive to the third stage. It can – at least for some learners – provide the amount of practice necessary for the gradual reduction of reaction time, error rate, and interference with other tasks that characterize the automatization process”. Similar cognitive perspectives might be applied to the classroom immersion context, on the assumption that it generates a ‘naturalistic’ academic context in which language is learnt through focusing on curricular content, one of the issues the monograph seeks to explore. 

As for the existing set of findings concerning how learners develop their target language abilities in ISLA, research has reached considerable consensus around some of the main issues by now, although some remain controversial, some barely examined, and some entirely unexplored. Let us now turn to a brief presentation of current thinking.  

Instructed SLA investigates L2 learning or acquisition that occurs as a result of teaching (Loewen, 2013, p. 2716). This field of research theoretically and empirically aims to understand “how the systematic manipulation of the mechanisms for learning and/or the conditions under which they occur enable or facilitate development and acquisition of a [second] language” (Loewen, 2015, p. 2). Formal instruction is a particular environment in instructed SLA that has been extensively researched for many decades. 

\citealt{In1988}, Michael Long reviewed eleven studies that examined the effect of FI on the rate and success of L2 acquisition. Of the studies that were reviewed, six of them showed that FI helped, three indicated that the instruction was of no help, and two produced ambiguous results. \citet{Long1983} claimed that instruction is beneficial to children and adults, to intermediate and advanced students, as well as in acquisition-rich and acquisition-poor environments. His final conclusion was that FI was more effective than “exposure-based” in L2 acquisition. These findings led researchers to ask whether instruction (FI) or exposure (SA, EMI, etc.) produced more rapid or higher levels of learning. 

Since \citegen{Long1988} seminal review of the effects of FI, there have been a number of studies of the effect of FI. For example, \citet{NorrisOrtega2001} conducted a meta-analysis of the effects of L2 instruction. Their study used a systematic procedure for research synthesis and meta-analysis to summarize findings from experimental and quasi-experimental studies between 1980 and 1998 that investigated the effectiveness of L2 instruction. Through their meta-analysis, they found that the literature suggests that instructional treatments are quite effective. They went on to investigate how effective instruction was when compared to simple exposure and found that there was still a large effect observed in favor of instructed learning.

Trenchs-\citet{Parera2009} conducted a study on the effects of FI and SA as it related to the acquisition of oral fluency. Her results found that although both contexts have different effects on oral fluency and production, both of these contexts did have a positive effect. She went on to say that “the differences between these two contexts [FI and SA] may not fulfill the popular expectation that SA makes learners produce more native-like speech than does FI at all levels” (p. 382). While these results do indicate that FI can have a positive effect on L2 acquisition, they are unable to demonstrate that FI has learning effects that are conclusively more positive than those of more naturalistic environments.

We now turn to the examination of the effects of SA, often contrasted with ISLA, and occasionally also with at-home immersion. SA research has generated a wealth of studies,  monographs, and handbooks on both sides of the Atlantic, starting in 1995 with Barbara \citegen{Freed1995} seminal publication, followed by, to name but a few, \citet{CollentineFreed2004a}, \citet{Pellegrino2005}, \citet{DufonChurchill2006}, De\citet{Keyser2007}, \citet{Collentine2009}, \citet{Kinginger2009}, \citet{Jackson2013}, \citet{LlanesMuñoz2013}, Regan, \citet{HowardLemée2009}, Mitchell, MacMannus \& Tracy-\citet{Ventura2015}, Pérez-Vidal (2014a; 2017),and \citet{SanzMorales2018}. Two periods can be distinguished in such research (Collentine, 2009; Pérez-Vidal, 2014b). The first one was initiated by Freed’s volume. In those years research mainly focused on the linguistic gains, or lack thereof, accrued with SA, with some attention paid to the impact of learner profiles and previous SA experiences (see for example, Brecht, Davidson \& Ginsberg, 1995). Following that, new themes, besides linguistic impact, and new angles to approach them, have emerged throughout the second period. Following \citegen{Collentine2009} tripartite distinction, such new themes include: (i) cognitive, psycholinguistic approaches looking into cognitive processing mechanisms displayed while abroad; (ii) sociolinguistic approaches analyzing input and interaction from a macro- and a micro-perspective; and, most centrally, (iii) sociocultural approaches derived from a paradigm shift from a language-centric (i.e., etic) approach to a learner-centric (i.e., emic) one \citep{Devlin2014}. As established in Pérez-\citet[341]{Vidal2017}, indeed, within the latter paradigm, and in order to focus on the learner and his/her immediate circumstances, SA research has recently begun to investigate non-linguistic individual differences which affect learning in such a context, “that is: (a) intercultural sensitivity and identity changes; (b) affects, such as foreign language anxiety (FLA) or willingness to communicate (WTC) and enjoyment; (c) social networks, particularly through the use of new technologies and social platforms, and their effect on linguistic practice”. Now, as De\citet[313]{Keyser2014} emphasizes, “a picture is beginning to emerge of what language development typically takes place [during SA] and what the main factors are that determine the large amount of variation found from one study to another”. 

Turning now to the positive effects of SA on learners’ linguistic progress, in a nutshell, empirical studies paint a blurred picture. They seem to show that SA does not always result in greater success than FI in ISLA -  some learners do manage to make significant linguistic progress while abroad, while others do not (De\citealt{Keyser2007}; \citealt{Collentine2009}; \citealt{Llanes2011}; Pérez-\citealt{Vidal2015b}; \citealt{Sanz2014}). In fact, what such results seem to prove, is the notorious variation in amount of progress made, which has often been attributed to the variation in learners’ ability to avail themselves of the opportunities for practice that a SA context offers. These differences in turn are explained by learners’ individual ability for self-regulation while abroad, as further discussed below \citet{UshiodaDörnyei2012}. 

Looking at progress in more detail, empirical research has repeatedly shown that oral production seems to be the winner, with effects on fluency being significantly positive after SA (Towell, \citealt{HawkinsBazergui1996}; Freed, \citealt{SegalowitzDewey2004}; \citealt{LlanesMuñoz2009}; Valls-\citealt{FerrerMora2014}). One interesting related finding has been made concerning the nature of the programmes \citep{Beattie2014}: robust immersion programmes organized at home and including a substantial number of hours of academic work on the part of the learners can be as beneficial as a similar length of time spent abroad (i.e., Freed, \citealt{SegalowitzDewey2004}). In contrast to the results for fluency in oral production, results for grammatical accuracy and complexity have been mixed, with De\citet{Keyser1991} not finding much improvement, whereas \citet{Howard2005} or Juan-Garau, Salazar-Noguera \& Prieto-\citet{Arranz2014}, to name but a few, report that progress is made after a period spent abroad. The other main area of improvement is pragmatics, in particular when associated with the use of formulaic routines, and perception and production of speech acts (see for a summary Pérez-\citealt{VidalShively2018}), and particularly when paired with pragmatics instruction. This takes us back to the key question of how the nature of the exchange programme can affect linguistic outcomes. More specifically, issues %
% DONE
%
%Consider mentioning these issues in the order in which they will be addressed (initial level, type of accommodation…)
such as type of accommodation, length of the stay, or initial level, have been found to significantly determine linguistic and cultural development while abroad. Concerning initial level, \citet{Collentine2009} stated that there should be a threshold level which learners must reach to benefit fully from the SA learning context. Once that level has been reached, most studies report better results for their respective lower level groups, confirming that the kind of practice most common while abroad, that is interaction in daily communication, mostly benefits the less advanced learners, while academic work done outside the classroom may benefit the most advanced ones \citep{Kinginger2009}. As for type of accommodation, home-stays with families have proved most beneficial An alternative option is with the so called \textit{family} \textit{learning} \textit{housing}, where students reside with target language speakers of their own age, having signed a language pledge not to use any other language but the target language \citep{Kinginger2015}. Length of stay also seems to be associated with advanced level learners, who may require longer periods to automatize the larger number of structures they have learnt at home than the lower level learners (De\citealt{Keyser2014}). However, interestingly, shorter periods abroad, of less than one month, may also significantly benefit EFL learners’ fluency, accuracy and listening abilities (L\citealt{LanesMuñoz2009}). Three month periods may be more beneficial than six months (Lara, Mora \& Pérez-\citealt{Vidal2016}).Listening has in fact clearly been shown to undergo significant progress while abroad (Beattie,Valls-Ferrer \& Pérez-\citealt{Vidal2014}), as has reading \citep{Dewey2004}. Writing and vocabulary have also been shown to significantly benefit from SA (\citealt{Sasaki2007}; 2011; \citealt{Barquin2012}; Zaytseva, Miralpeix \& Pérez-\citealt{Vidal2018}).

Regarding learners’ individual differences, age seems to play a role, as SA has been shown to be more beneficial for children than for adults in relative terms (\citealt{LlanesMuñoz2013}). Regarding aptitude, a certain level of working memory (\citealt{SundermanKroll2009}), phonological memory (O’Brien, Segalowitz, \citealt{FreedCollentine2007}) and processing speed \citep{Taguchi2008} seem to correlate with accurate L2 production, oral production and reception of pragmatic intentions, respectively. Finally, concerning the emotional variables underlying self-regulation during exchanges in the target language country, the expectation is that motivation will have a positive role and that anxiety, paired with the capacity for enjoyment, will as well. Dewaele, \citet{ComanaruFaraco2015} have found that SA benefits emotional stability, self-confidence and resourcefulness.  While identity goes through a process of repositioning, this process is not exempt from difficulties, which often conditions degree of contact with target language speakers while abroad. More willingness to communicate and less foreign language anxiety seem to obtain during SA (\citealt{DewaeleWei2013}; Dewaele, \citealt{ComanaruFaraco2015}). 

Turning to the third type of context, although it is still in its infancy, immersion, the integration of content and language as an educational approach in primary, secondary (CLIL) and tertiary levels (ICL), has also given rise to a sizeable number of research studies (such as for example: Admiraal, Westhof \& De \citealt{Boot2006}; Dalton-\citealt{Puffer2008}; \citealt{Airey2012}; Cenoz, \citealt{GeneseeGörter2014}). The integration of content and language in higher education (ICLHE) came to be recognized in its own right in 2004, with the first conference examining this context, and has steadily grown to this day \citep{Wilkinson2004}. 

Findings from immersion and CLIL contexts, abundantly examined in the SLA literature, report that CLIL and immersion learners demonstrate language gains superior to learners who participate in FI alone, with equal or superior content learning outcomes (\citealt{WescheSkehan2002}; \citealt{Genesee2004}; Jiménez Catalán, Ruiz de \citealt{ZarobeCenoz2006}; Seikkula-\citealt{Leino2007}). Specifically, gains are reported in receptive skills, vocabulary, morphology, and fluency, whereas fewer gains have been observed according to syntax, writing, pronunciation and pragmatics (Dalton-\citealt{Puffer2008}), although results may be mixed (Pérez-\citealt{VidalRoquet2014}). Research on non-linguistic outcomes has found that CLIL learners seem to be more motivated, or that CLIL can maintain students’ interests and change attitudes towards multilingualism. Moreover, students generally perceive CLIL participation as a positive experience (\citealt{LasagabasterSierra2009}). 

Turning now to adult education, the main focus of this monograph, a large body of research has been generated within the frame of ICLHE which is specifically interested in the widespread implementation of English-taught programs at mainly post graduate levels. This has come to be known as English medium instruction (EMI) which is characterized as a setting where English is used as a medium for instruction by, and for non-native English speakers in non-English speaking environments (\citealt{HellekjaerHellekjaer2015}). Researchers in this field have begun investigating the phenomenon from a wide variety of angles, for example by looking at the implementation and policy making end of the spectrum \citep{Tudor2007}. What has been found is that the implementation of EMI must be carefully managed in order not to create tensions, considering the role of the first language, attitudes towards English, and the widespread effects of internationalization, not only affecting faculty and students, but also governing bodies and administration (Doiz, \citealt{LasagabasterSierra2014}). Others report on beliefs, attitudes and challenges from both the student/learner perspective and the faculty/institution’s perspective. Findings show that stakeholders in EMI relate English instruction to internationalization very clearly, with some believing that one cannot exist without the other (\citealt{HenryGoddard2015}). This belief also proves to be a strong motivator for students to enroll in EMI courses (Margić \& Žeželić 2015), although the experience does not always meet their expectations regarding language improvement and more support is often desired \citep{Sert2008}. Finally, perhaps the least investigated aspect of EMI involves the assessment of outcomes measured in linguistic as well as non-linguistic terms. 

On the one hand there are investigations looking at non-linguistic effects from EMI participation (\citealt{Gao2008}; González \citealt{Ardeo2016}). Research shows that a gradual implementation supporting both faculty and students is the most effective for maintaining and creating positive attitudes and motivation (\citealt{ChenKraklow2015}). On the other hand, there are studies regarding the content learning implications of learning through a foreign language \citep{Dafouz2014}. It has been argued that upon completion of a degree program there is no difference in content knowledge (Dafouz \& Camacho-\citealt{Miñano2016}). A few studies investigating language outcomes from such a context (\citealt{LeiHu2014}; Ament \& Pérez \citealt{Vidal2015}; \citealt{Ritcher2017}) show little evidence of language improvement from EMI participation. They also reveal that at this point there is simply not enough research to point to any clear conclusions. EMI is growing rapidly around the world and its close relationship with internationalization will ensure its continuance for time to come. What must be kept in mind is that, in order to properly implement, benefit from, and provide appropriate support to faculty and institutions offering EMI instruction, and maintain quality education, more research on this context must be carried out, specifically considering both linguistic and non-linguistic effects, which is precisely what this monograph aims to bring to light. 

However, to our knowledge, no publication exists which places the three contexts along the continuum already mentioned, as suggested in Pérez-Vidal (2011; 2014b) with SA as ‘the most naturalistic’ context on one extreme, ISLA on the other, and ICL somewhere in between. The present monograph seeks to make a first attempt at filling such a gap, by including a number of studies analysing the effects of EMI, and another series of studies doing the same with SA, in contrast with ISLA. In such a comparison it is further assumed that EMI programmes are often experienced at the home institution either as an ‘international experience at home’ (internationalization at home), or as a preparation for the ‘real’ experience of an SA period spent in the target language country, in which learners will most probably be expected to regularly attend academic courses. In such a circumstance, whatever the local language, quite probably some of the courses offered, if not all, will be EMI courses for international students, that is, they will be what we call ‘international classrooms’ (\citealt{Coleman2013}; \citealt{Leask2015}). 

The monograph will thus be organized around the two contexts, EMI and SA, on the understanding that their effects will be contrasted with those obtained in ISLA, when appropriate. Both linguistic and non-linguistic phenomena will be investigated, employing quantitative but also qualitative methods, independently or combined. Regarding target countries in the immersion programmes examined, they include data from Spain and Colombia. Of the SA programmes scrutinized, data include exchanges having the following destinations: England, Ireland, France, Germany and Spain, in Europe, but also Canada, the USA, China, Brazil and Australia. The EMI chapters deal with tertiary level language learners, a section of the population which has received much less attention in research thus far, compared to secondary or primary learners, as mentioned above. Similarly, one SA chapter deals with adolescent learners, again a research population scarcely examined in such a context.

As for the internal organization of the volume, following the introduction by the editors, the first chapters will deal with EMI contexts of acquisition, and the remaining ones with SA contexts. 

More specifically, we open up the monograph with four chapters devoted to the immersion context: three examine tertiary education data, and the last one primary and secondary. In Chapter 1, Dakota Thomas-Wilhelm and Carmen Pérez Vidal explore EMI in Catalonia, Spain, in contrast with ISLA, focusing on a syntactic phenomenon and its cognitive correlates, namely English countable and uncountable nouns. In Chapter 2, Jennifer Ament and Júlia Barón examine two EMI programmes with different intensity, also in Catalonia, looking into pragmatics, namely, the use of English discourse markers and their acquisition in the EMI context. Chapter 3, by Sofia Moratinos- Johnston, Maria Juan-Garau and Joana Salazar-Noguera, analyses a non-linguistic issue, that is, learners’ linguistic self-confidence and perceived level of English according to the number of EMI subjects taken at university in the Balearic Islands, Spain. Chapter 4, by Isabel-Tejada Sánchez and Carmen Pérez-Vidal, closes the set of chapters devoted to immersion, by investigating the complexity, accuracy and fluency of written productions by young EFL immersion learners in Colombia.

Subsequently, the series of chapters on SA begins with Chapter 5, by Pilar Avello, which takes a fresh perspective and discusses the methodological intricacies associated with the measurement and analysis of pronunciation gains obtained during a sojourn abroad in an English-speaking country (England, Ireland, Canada, the USA, Australia). Chapter 6 by Victoria Monge and Angelica Carlet, contrasts ISLA and SA. These authors compare L2 phonological development, following a three-month period in any of the above-mentioned English-speaking countries, while controlling for proficiency level, in an attempt to follow up on \citegen{Mora2008} seminal study with a reverse design. In Chapter 7 Carmen del Rio, Maria Juan-Garau and Carmen Pérez-Vidal contrast the impact of a three-month SA period and FI at home, in the case of adolescent EFL learners, an age band which has received comparatively less attention than others, focusing on the learners’ foreign accent and comprehensibility, as judged by a group of non-native listeners, with the objective of assessing progress, following \citet{TrofimovichIsaacs2012}. \textstylenormaltextrun{Motivation, identity and international posture is the focus of Chapter 8, in which Leah Geoghegan compares tertiary level students spending a SA in an English-speaking country with those in Germany or France, using qualitative research tools in order to gain a more detailed picture of the role of ELF in SA. After that, Chapter 9 by Iryna Pogorelova and Mireia Trenchs explore} intercultural adaptation during the experience of a SA period in different countries in Europe, but also in Canada, the USA, China, Brazil, and Australia. Finally, in Chapter 10 Ariadna Sánchez Hernández deals with acculturation and pragmatic learning by international students in the USA, to close the series of chapters dealing with SA. 

\section{ References}

Admiraal, Wilfred, Gerard Westhoff \& Kees de Bot. 2006. Evaluation of bilingual secondary education in the Netherlands: Student’s language proficiency in English. \textit{Educational} \textit{Research} \textit{and} \textit{Evaluation} 12(1). 75-93.  

Airey, John. 2012. I don’t teach language.The linguistic attitudes of physics lecturers in Sweden. \textit{AILA} \textit{Review}{} 25. 64-79.

Ament, Jennifer \& Carmen Pérez Vidal. 2015. Linguistic outcomes of English medium instruction programmes in higher education: A study on Economics undergraduates at a Catalan University.\textit{Higher} \textit{Learning} \textit{Research} \textit{Communications} 5(1). 47-68.

Barquin, Elisa. 2012. \textit{Writing} \textit{development} \textit{in} \textit{a} \textit{study} \textit{abroad} \textit{context}. Unpublished Dissertation. Barcelona: Universitat Pompeu Fabra.

Beattie, John. 2014. The ‘ins and outs’ of a study abroad programme: The SALA exchange programme. In Carmen Pérez-Vidal (ed.), \textit{Language} \textit{acquisition} \textit{in} \textit{study} \textit{abroad} \textit{and} \textit{formal} \textit{instruction} \textit{contexts,} 59-87. Amsterdam: John Benjamins Publishing. 

Beattie, John, Margalida Valls-Ferrer \& Carmen Pérez-Vidal. 2014. Listening performance and onset level in formal instruction and study abroad. In Carmen Pérez-Vidal (ed.), \textit{Language} \textit{acquisition} \textit{in} \textit{a} \textit{study} \textit{abroad} \textit{and} \textit{formal} \textit{instruction} \textit{contexts,} 195-217. Amsterdam: John Benjamins Publishing. 

Björkman, Beyza. 2013. \textit{English} \textit{as} \textit{an} \textit{academic} \textit{lingua} \textit{franca:} \textit{an} \textit{investigation} \textit{of} \textit{form} \textit{and} \textit{communicative} \textit{effectiveness.} Boston/Berlin: Walter de Gruyter. 

Brecht, Richard, Dan Davidson \& Ralph Ginsberg. 1995. Predictors of foreign language gain during study abroad. In Barbara Freed (ed.), \textit{Second} \textit{language} \textit{acquisition} \textit{in} \textit{a} \textit{study} \textit{abroad} \textit{context}, 37–66. Amsterdam: John Benjamins Publishing. 

Cenoz, Jason, Fred Genesee \& Derek Görter. 2014. Critical analysis of CLIL: Taking stock and looking forward. \textit{Applied} \textit{linguistics} 35(3). 243-262.

Chen, Yih, Lan Ellen \& Deborah Kraklow. 2015. Taiwanese College Students’ Motivation and Engagement for English Learning in the Context of Internationalization at Home: A Comparison of Students in EMI and Non-EMI Programs. \textit{Journal} \textit{of} \textit{Studies} \textit{in} \textit{International} \textit{Education} 19(1). 46–64. doi:10.1177/1028315314533607.

Coleman, Jim. 2013. English-Medium teaching in European Higher Education. \textit{Language} \textit{Teaching} 39 (1). 1-14.

Coleman, Jim. 2015. Social circles during residence abroad: What students do, and who with. In Ross Mitchell, Nicole Tracy-Ventura \& Kevin McManus (eds.), \textit{Social} \textit{interaction,} \textit{identity} \textit{and} \textit{language} \textit{learning} \textit{during} \textit{residence} \textit{abroad}, 33-50. Eurosla Monograph Series 5.

Collentine, John. 2009. Study abroad research: Findings, implications, and future directions. In Michael. H. Long \& Catherine Doughty (eds.), \textit{The} \textit{handbook} \textit{of} \textit{second} \textit{language} \textit{teaching}, 218-233. Malden, MA: Blackwell.

Collentine, John. \& Barbara Freed. 2004a. \textit{Studies} \textit{in} \textit{Second} \textit{Language} \textit{Acquisition} 26(2).

Collentine, John. \& Barbara Freed. 2004b. Introduction: Learning context and its effects on second language acquisition. \textit{Studies} \textit{in} \textit{Second} \textit{Language} \textit{Acquisition} 26(2), 153-171.

Commission of the European Communities. 1995. \textit{Teaching} \textit{and} \textit{learning:} \textit{Towards} \textit{a} \textit{learning} \textit{society.} \textit{449} \textit{White} \textit{Paper} \textit{on} \textit{Education} \textit{and} \textit{Learning}. Brussels: DGV.

Commission of the European Communities. 2013. \textit{On} \textit{the} \textit{way} \textit{to} \textit{ERASMUS+.} \textit{A} \textit{statistical} \textit{overview} \textit{of} \textit{the} \textit{ERASMUS} \textit{programme} \textit{in} \textit{2011-12.} Retrieved from \url{http://ec.europa.eu/education/}

\begin{styletextbox}
Dafouz, Emma. 2014. Integrating content and language in European higher education: An overview of recurrent research concerns and pending issues. In Psaltou-Joycey Agathopoulou \& Marina Mattheoudakis (eds.), \textit{Cross-Curricular} \textit{Approaches} \textit{to} \textit{Language} Education. 289-304. Cambridge: Cambridge Scholars. 
\end{styletextbox}

Dafouz, Emma \& Maria Camacho-Miñano. 2016 Exploring the impact of English-medium instruction on university student academic achievement: The case of accounting. \textit{English} \textit{for} \textit{Specific} \textit{Purposes} 44. 57–67

Dalton-Puffer, Christiane. 2008. Communicative Competence in ELT and CLIL Classrooms: Same or Different. \textit{Views.} \textit{Vienna} \textit{English} \textit{Working} \textit{Papers} 17(3). 14-21

DeKeyser, Robert. 1991. Foreign language development during a semester abroad. In Barbara Freed (ed.), \textit{Foreign} \textit{language} \textit{acquisition:} \textit{Research} \textit{and} \textit{the} \textit{classroom,} 104-119. Lexington M.A.: D. C. Heath.

DeKeyser, Robert. 2007. Study abroad as foreign language practice. In Robert DeKeyser (ed.), \textit{Practicing} \textit{in} \textit{a} \textit{second} \textit{language:} \textit{Perspectives} \textit{from} \textit{applied} \textit{linguistics} \textit{and} \textit{cognitive} \textit{psychology}, 208-226. Cambridge: Cambridge University Press.

DeKeyser, Robert. 2014. Research on language development during study abroad: methodological considerations and future perspectives. In Carmen Pérez-Vidal (ed.), \textit{Language} \textit{acquisition} \textit{in} \textit{study} \textit{abroad} \textit{and} \textit{formal} \textit{instruction} \textit{contexts}, 313-327. Amsterdam/Philadelphia: John Benjamins Publishing.

Devlin, Anne Marie. 2014. \textit{The} \textit{impact} \textit{of} \textit{study} \textit{abroad} \textit{on} \textit{the} \textit{acquisition} \textit{of} \textit{sociopragmatic} \textit{variation} \textit{patterns:} \textit{The} \textit{case} \textit{of} \textit{non-native} \textit{speaker} \textit{English} \textit{teachers}. Bern: Peter Lang. 

Dewaele, Jean Marc. 2007. The effect of multilingualism, sociobiographical and situational factors on communicative anxiety and foreign language anxiety of mature language learners. \textit{The} \textit{International} \textit{Journal} \textit{of} \textit{Bilingualism} 11(4). 391-409.

Dewaele, Jean Marc, Ruxandra.S. Comanaru, \& Martin Faraco. 2015. The affective benefits of a pre-sessional course at the start of study abroad. In Ros Mitchell, Nicole Tracy-Ventura \& Kevin McManus (eds.), \textit{Social} \textit{interaction,} \textit{identity} \textit{and} \textit{language} \textit{learning} \textit{during} \textit{residence} \textit{abroad,} 33-50. Eurosla Monograph Series 5.

Dewaele, Jean-Marc \& Li Wei. 2013. Is multilingualism linked to a higher tolerance of ambiguity? \textit{Bilingualism:} \textit{Language} \textit{\&} \textit{Cognition} 16(1). 231.

Dewey, Dan. 2004. A comparison of reading development by learners of Japanese in intensive domestic immersion and study abroad contexts. \textit{Studies} \textit{in} \textit{Second} \textit{Language} \textit{Acquisition} 26(2), 303-327.

Doiz, Aintzane, David Lasagabaster, \& Josep Maria Sierra. 2014 Language friction and multilingual policies in higher education: the stakeholders' view, \textit{Journal} \textit{of} \textit{Multilingual} \textit{and} \textit{Multicultural} \textit{Development} 35(4). 345-360.

DuFon, Margaret, A. \& Eton Churchill (eds.). 2006. \textit{Language} \textit{learners} \textit{in} \textit{study} \textit{abroad} \textit{contexts}. Clevedon: Multilingual Matters.

Freed, Barbara. 1995. Second language acquisition in study abroad contexts. Amsterdam/Philadelphia: John Benjamins. 

Freed, Barbara, Fred Segalowitz, \& Dan Dewey. 2004. Context of learning and second language fluency in French: Comparing regular classroom, study abroad and intensive domestic immersion programs. \textit{Studies} \textit{in} \textit{Second} \textit{Language} \textit{Acquisition} 26(2), 275-301.

Gao, Xuesong. 2008. Shifting motivational discourses among mainland Chinese students in an English medium tertiary institution in Hong Kong: A longitudinal inquiry. \textit{Studies} \textit{in} \textit{Higher} \textit{Education} 33(5). 599–614. doi:10.1080/03075070802373107.

Genesee, Fred. 2004. What do we know about bilingual education for majority language students. In Fred Genesee, Tessa Bathia and William. Ritchie (eds.), \textit{Handbook} \textit{of} \textit{bilingualism} \textit{and} \textit{multiculturalism}, 547-576. Malden MA: Blackwell Publishing

González Ardeo, Mikel Joseba. 2016. Engineering student’s instrumental motivation and positve attitude towards learning English in a trilingual tertiary setting. \textit{Ibérica} 32. 179–200.

Hellekjaer, Glen Ole \& Anne-Inger Hellekjaer. 2015. From tool to target language: Arguing the need to enhance language learning in English-medium instruction courses and programs. In Dimova, Slobodanka, Hultgren, Anna Kristina, \& Christian Jensen (eds.), \textit{English-Medium} \textit{Instruction} \textit{in} \textit{European} \textit{Higher} \textit{Education}, 317-324. Berlin: Walter de Gruyter. 

Henry, Alastair \& Angela Goddard. 2015. Bicultural or Hybrid? The Second Language Identities of Students on an English-Mediated University Program in Sweden. \textit{Journal} \textit{of} \textit{Language,} \textit{Identity} \textit{and} \textit{Education} 14(4). 255–274. doi:10.1080/15348458.2015.1070596.

Hernández, Todd A. 2010. Promoting speaking proficiency through motivation and interaction: The study abroad and classroom learning contexts. \textit{Foreign} \textit{Language} \textit{Annals} 43(4). 650–670.

House, Juliana. 2013. Developing pragmatic competence in English as a lingua franca: Using discourse markers to express (inter)-subjectivity and connectivity. \textit{Journal} \textit{of} \textit{Pragmatics} 59(A). 57-67.

Howard, Martin. 2005. Second language acquisition in a study abroad context: A comparative investigation of the effects of study abroad and formal language instruction on the L2 learner’s grammatical development. In Alex Housen \& Michael Pierrard (eds.), \textit{Investigations} \textit{in} \textit{instructed} \textit{second} \textit{language} \textit{acquisition}, 495-530. Berlin: Mouton de Gruyter.

Jackson, Jane. 2013. Pragmatic development in study abroad contexts. In Carol.A Chapelle (ed.), \textit{The} \textit{Encyclopedia} \textit{of} \textit{Applied} \textit{Linguistics}, 1- 12. Hoboken: NJ: Wiley-Blackwell. 

Jiménez Catalán, Rosa María, Yolanda Ruiz de Zarobe \& Jasone Cenoz Iragui. 2006. Vocabulary profiles of English foreign language learners in English as a subject and as a vehicular language. \textit{Views.} \textit{Vienna} \textit{English} \textit{Working} \textit{Papers}: 15 (3): 23-27. 

Juan-Garau, Maria., Joana Salazar-Noguera, \& José I. Prieto-Arranz. 2014. English L2 learners’lexico-grammatical and motivational development at home and abroad. In Carmen Pérez-Vidal (ed.), \textit{Language} \textit{Acquisition} \textit{in} \textit{Study} \textit{Abroad} \textit{and} \textit{Formal} \textit{Instruction} \textit{Contexts}, 259-283. Amsterdam/Philadelphia: John Benjamins Publishing. 

Kinginger, Celeste. 2009. \textit{Language} \textit{learning} \textit{and} \textit{study} \textit{abroad.} \textit{A} \textit{critical} \textit{reading} \textit{of} \textit{research.} Houndmills: Palgrave: McMillian.

Kinginger, Celeste. 2015. Language socialization in the homestay: American high school students in China. In Ros Mitchell, Nicole Tracy-Ventura \& Kevin McManus (eds.), \textit{Social} \textit{interaction,} \textit{identity} \textit{and} \textit{language} \textit{learning} \textit{during} \textit{residence} abroad, 33-53. Eurosla Monograph Series 5.

Lara, Anne R., Joan Carles Mora, \& Carmen Pérez-Vidal. 2014. How long is long enough? L2 English Development through study abroad programmes varying in duration. \textit{Innovation} \textit{in} \textit{Language} \textit{Learning} \textit{and} \textit{Teaching} 9(1). 1-12. 

Lasagabaster, David \& Juán Manuel Doiz Aintzane. (eds.). 2014. \textit{Motivation} \textit{and} \textit{foreign} \textit{language} \textit{learning:} \textit{From} \textit{theory} \textit{to} \textit{practice}. New York/Amsterdam: John Benjamins. 

Lasagabaster, David, \& Josep. Maria Sierra. 2009. Language attitudes in CLIL and traditional EFL classes. \textit{International} \textit{CLIL} \textit{Research} \textit{Journal} 1(2). 4-17.

Leask, Betty. 2015. \textit{Internationalizing} \textit{the} \textit{Curriculum}. New York: Routledge.

Lei, Jun \& Guangwei Hu. 2014. Is English-medium instruction effective in improving Chinese undergraduate students’ English competence? \textit{IRAL} 52(2). 99-126.

Llanes, Angels.2011. The many facets of study abroad: an update of the research on L2 gains emerged during a SA experience. \textit{International} \textit{Journal} \textit{of} \textit{Multilingualism} 8(3).189-215. 

LLanes, Angels, \& Carmen Muñoz. 2009. A short stay abroad: Does it make a difference? \textit{System} 37. 353-365.

LLanes, Angels, \& Carmen Muñoz. 2013. Age effects in a study abroad context: Children and adults studying abroad and at home. \textit{Language} \textit{Learning} 64(1). 1-28.

Leonard,  Karen R. \& Shea, Christine, E. L2 speaking development during study abroad: Fluency, accuracy, complexity and underlying cognitive factors. \textit{The} \textit{Modern} \textit{Language} \textit{Journal} 101(1). 179-193.

Loewen, Shawn. 2013. Instructed second language acquisition. In Carol A. Chapelle (ed.), \textit{The} \textit{encyclopedia} \textit{of} \textit{applied} \textit{linguistics}, 2716-2718. Malden, MA: Blackwell Publishing.

Loewen, Shawn. 2015. \textit{Instructed} \textit{Second} \textit{Language} \textit{Acquisition.} New York: Routledge.

Long, Michael. 1983. Does second language instruction make a difference? A review of research. \textit{TESOL} \textit{Quarterly} 17. 359-382.

Long, Michael. 1988. Instructed interlanguage development. In Leslie M. Beebe (ed.) \textit{Issues} \textit{in} \textit{second} \textit{language} \textit{acquisition:} \textit{Multiple} \textit{perspectives}, 115-141. Cambridge, MA: Newbury House Publishers. 

Margić, Branka. D \& Tea Žeželić. 2015. The implementation of English-medium instruction in Croatian higher education: Attitudes, expectations and concerns.{~In Ramón Plo Alanstrué \& Carmen Pérez-Llantada. (eds.),} \textit{English} \textit{as} \textit{a} \textit{scientific} \textit{and} \textit{research} \textit{language}, 311-332. Berlin: De Gruyter. 

Mitchell, Ros, Kevin McManus \& Nicole Tracy-Ventura. 2015. Placement type and language learning during residence abroad. In Ros Mitchell, Nicole Tracy-Ventura \& Kevin McManus (eds.), \textit{Social} \textit{interaction,} \textit{identity} \textit{and} \textit{language} \textit{learning} \textit{during} \textit{residence} abroad, 95-115. Eurosla Monograph Series 5.

Norris, John \& Lourdes Ortega. 2001. Does type of instruction make a difference? Substantive findings from a meta-analytic review. \textit{Language} \textit{Learning} 51. 157-213.

O’Brien, Irena, Norman Segalowitz, Barbara Freed \& Joseph Collentine. 2007. Phonological memory predicts second language fluency gains. \textit{Studies} \textit{in} \textit{Second} \textit{Language} \textit{Acquisition} (29)4. 557-581. 

Pellegrino, Valerie A. 2005. \textit{Study} \textit{abroad} \textit{and} \textit{second} \textit{language} \textit{use:} \textit{Constructing} \textit{the} \textit{self.} Cambridge: Cambridge University Press.

Pérez-Vidal, Carmen. 2011. Language acquisition in three different contexts of learning: Formal instruction, study abroad and semi-immersion (CLIL). In Yolanda Ruiz de Zarobe, José María Sierra \& Francisco Gallardo del Puerto (eds.), \textit{Content} \textit{and} \textit{foreign} \textit{language} \textit{integrated} \textit{learning:} \textit{Contributions} \textit{to} \textit{multilingualism} \textit{in} \textit{European} \textit{contexts,} 103-129. Bern: Peter Lang.

Pérez-Vidal, Carmen. 2014a. \textit{Second} \textit{language} \textit{acquisition} \textit{in} \textit{study} \textit{abroad} \textit{and} \textit{formal} \textit{instruction} \textit{contexts}. Amsterdam: John Benjamins.

Pérez-Vidal, Carmen. 2014b. Study abroad and formal instruction contrasted: The SALA project.’ In Carmen Pérez Vidal (ed.), \textit{Second} \textit{language} \textit{acquisition} \textit{in} \textit{study} \textit{abroad} \textit{and} \textit{formal} \textit{instruction} \textit{contexts}, 17-57. Amsterdam: John Benjamins.

Pérez-Vidal, Carmen. 2015a. Languages for all in education: CLIL and ICLHE at the crossroads of multilingualism, mobility and internationalization. In Maria Juan-Garau \& Joana Salazar-Noguera (eds.), \textit{Content-based} \textit{learning} \textit{in} \textit{multilingual} \textit{educational} \textit{environments}, 31-51. Berlin: Springer.

Pérez-Vidal, Carmen. 2015b. Practice makes best: Contrasting learning contexts, comparing learner progress. \textit{International} \textit{Journal} \textit{of} \textit{Multilingualism} 12(4). 453-470.

Pérez-Vidal, Carmen. 2017. Study abroad and ISLA. In Shawn Loewen \& Masatoshi Sato (eds.), \textit{The} \textit{Routledge} \textit{handbook} \textit{of} \textit{study} \textit{abroad} \textit{research} \textit{and} \textit{practice}, 339-361. London: Routledge: Taylor \& Francis Group.

Pérez-Vidal, Carmen, Neus Lorenzo \& Mireia Trenchs. 2017. Una nova mirada a les llengües en l’educació: el plurilingüisme i l’internacionalització. In Josep M. Vilalta (ed.), \textit{Reptes} \textit{de} \textit{l’Educació} \textit{a} \textit{Catalunya:} \textit{\citealt{Anuari2015}}, 139-195. Barcelona: Fundació Jaume Bofill i Edicions el Llum.

Pérez-Vidal, Carmen \& Helena Roquet. 2014. CLIL in context: Profiling language abilities. In Maria Juan-Garau \& Joana Salazar-Noguera (eds.), \textit{Content-Based} \textit{Language} \textit{Learning} \textit{in} \textit{Multilingual} \textit{Educational} \textit{Environments,} 237-254. Berlin: Springer.

Kinginger, Celeste. (2009). \textit{Language} \textit{learning} \textit{and} \textit{study} \textit{abroad.} \textit{A} \textit{critical} \textit{reading} \textit{of} \textit{research.} Hampshire\textit{:} Palgrave MacMillan.

Pérez-Vidal, Carmen \& Rachel Shively. (forthcoming). Pragmatic development in study abroad settings. In Nakoto Taguchi (ed.) \textit{The} \textit{Routledge} \textit{handbook} \textit{of} \textit{study} \textit{abroad} \textit{research} \textit{and} \textit{practice}. New York: Routledge.

Regan, Vera, Martin Howard \& Isabelle Lemée. 2009. \textit{The} \textit{acquisition} \textit{of} \textit{sociolinguistic} \textit{competence} \textit{in} \textit{a} \textit{study} \textit{abroad} \textit{context.} Clevedon: Multilingual Matters.

Ritcher, Karin. 2017. Researching tertiary EMI and pronunciation. A case from Vienna. In Jennifer Valcke \& Robert Wilkinson (eds.), \textit{Integrating} \textit{Content} \textit{and} \textit{Language} \textit{in} \textit{Higher} \textit{Education;} \textit{Perspectives} \textit{on} \textit{professional} \textit{practice}, 117-134. Frankfurt: Peter Lang. 

Sanz, Cristina. 2014. Contribution of study abroad research to our understanding of SLA processes and outcomes: The SALA project, an appraisal. In Carmen Pérez-Vidal (ed.), \textit{Language} \textit{acquisition} \textit{in} \textit{study} \textit{abroad} \textit{and} \textit{formal} \textit{instruction} \textit{contexts}, 1-17. Amsterdam/Philadelphia: John Benjamins.

Sanz, Cristina \& Alfonso Morales-Front, (eds.) 2018. \textit{The} \textit{Routledge} \textit{Handbook} \textit{of} \textit{Study} \textit{Abroad} \textit{Research} \textit{and} \textit{Practice.} New York: Routledge. 

Sasaki, Mitsuko. 2007. Effects of study-abroad experiences on EFL writers: A multiple-data analysis. \textit{The} \textit{Modern} \textit{Language} \textit{Journal} 91. 602–620.

Sasaki, Mitsuko. 2011. Effects of varying lengths of study-abroad experiences on Japanese EFL students’ L2 writing ability and motivation: A longitudinal study. \textit{TESOL} \textit{Quarterly} 45\textit{(1)}. 81-105.

Seikkula-Leino, Jaana. 2007. CLIL learning: Achievement levels and affective factors. \textit{Language} \textit{and} \textit{Education} 21(4). 328-341.

Sert, Nehir. 2008. The language of instruction dilemma in the Turkish context.  \textit{System} 36. 156-171.

Sunderman, Gretchen \& Judith Kroll. 2009. When study abroad experience fails to deliver: The internal resources threshold effect. \textit{Applied} \textit{Psycholinguistics} 30. 1-21.

Taguchi, Naoko. 2008. Cognition, language contact, and the development of pragmatic comprehension in a study-abroad context. \textit{Language} \textit{Learning} 58(1). 33-71. 

Taguchi, Naoko, Feng Xiao \& Shuai Li. 2016. Effects of intercultural competence and social contact on speech act production in a Chinese study abroad context. \textit{The} \textit{Modern} \textit{Language} \textit{Journal} 100. 775-796.

Towell, Richard, Roger Hawkins \& Nives Bazergui. 1996. The development of fluency in advanced learners of French. \textit{Applied} \textit{Linguistics,} \textit{17}(1). 84-119. 

Trenchs-Parera, Mireia. 2009. Effects of formal instruction and study abroad on the acquisition of native-like oral fluency. \textit{Canadian} \textit{Modern} \textit{Language} \textit{Review} 65(3). 365-393.

Ushioda, Ema \& Zoltan Dörnyei. 2012. Motivation. In Susan. Gass \& Allison Mackey (eds.), \emph{The} \emph{Routledge} \emph{handbook} \emph{of} \emph{second} \emph{language} \emph{acquisition}{,} 96-409. New York: Routledge.

Valls-Ferrer, Margalida \& Mora Joan Carles. 2014. L2 fluency development in formal instruction and study abroad: The role of initial fluency level and language contact. In Carmen Pérez-Vidal (ed.), \textit{Language} \textit{Acquisition} \textit{in} \textit{Study} \textit{Abroad} \textit{and} \textit{Formal} \textit{Instruction} \textit{Contexts,} 111-137. Amsterdam/Philadelphia: John c, content-based instruction, and task-based learning. ~In Robert Kaplan ed. \textit{Handbook} \textit{of} \textit{applied} \textit{linguistics.} Oxford: Oxford University Press. 207-228.

Wächter, Bern \& Maiwörm, Friedhelm (eds.).  2014. \textit{English-taught} \textit{programmes} \textit{in} \textit{European} \textit{higher} \textit{education:} \textit{The} \textit{state} \textit{of} \textit{play} \textit{in} \textit{2014.} Lemmens Medien GmbH.

Wilkinson, Robert (ed). 2004. \textit{Integrating} \textit{Content} \textit{and} \textit{Language:} \textit{Meeting} \textit{the} \textit{challenge} \textit{of} \textit{a} \textit{multilingual} \textit{higher} \textit{education}. Maastricht: Universitaire Pers Masstricht. 

Zaytseva, Victoria \& Miralpeix,Imma \& Pérez-Vidal, Carmen. 2018. Vocabulary acquisition during study abroad: A comprehensive review of research. In Sanz, Cristina \& Alfonso Morales-Front, (eds.) 2018. \textit{The} \textit{Routledge} \textit{Handbook} \textit{of} \textit{Study} \textit{Abroad} \textit{Research} \textit{and} Practice, 210-224. New York: Routledge. 


\begin{verbatim}%%move bib entries to  localbibliography.bib
\end{verbatim}