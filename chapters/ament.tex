\documentclass[output=paper]{langsci/langscibook} 
\ChapterDOI{10.5281/zenodo.1300612}
\author{Jennifer Ament\affiliation{Universitat Pompeu Fabra}\lastand 	Júlia Barón Parés\affiliation{Universitat Internacional de Catalunya; Universitat de Barcelona}}  
 
\title{The acquisition of discourse markers in the English-medium instruction context}
\shorttitlerunninghead{The acquisition of discourse markers}
\abstract{This study focuses on the effects of the context of learning on language acquisition by comparing the production of discourse markers (DMs) in oral output of English-medium instruction (EMI) students (N = 7) with non-EMI students (N = 9). Data were elicited through an oral discourse completion task and a conversation task. Four types of DMs were identified: cognitive, interpersonal, structural and referential. Quantitative analysis reveals that EMI students tend to produce longer responses and more structural markers, as opposed to control students, who use more referential markers.  A qualitative interpretation of the data suggests that the EMI participants mark their discourse for their own as well as for their interlocutor’s benefit, specifically by using structural markers to ensure clear interpretation of utterances. The study further suggests that participation in an EMI program may lead to pragmatic benefits specifically in terms of the type and quality of DMs used, rather than of their frequency and overall variety.  However, the study also indicates that this context alone may not be sufficient for the acquisition of all types of markers, and that there are many other factors at play in the acquisition of this pragmatic feature.}
\maketitle

\begin{document}   

\section{Introduction}

This exploratory study examines the \isi{acquisition} of discourse markers (\isi{DMs}) in second language \isi{acquisition}. The function of \isi{DMs} as connectors in discourse makes them essential to smooth communication, as they facilitate the correct interpretation of an utterance and express the speakers’ intentions \citep{Ariel1998}. Despite the attested importance of these markers, \isi{DMs} are seldom addressed in second language \isi{classroom instruction} \citep{Vellenga2004}. Thus, learners are left with the difficult task of, firstly, interpreting, and secondly, integrating them effectively into their own speech. As this volume highlights, the context of learning plays an important role in second language \isi{acquisition}, for it has been found that different contexts of learning foster the development of different language skills. General conclusions from research are that, for optimal language learning to occur, participation in more than one context is desirable (\citealt{Pérez-Vidal2014}). More particularly, \isi{integrated content} and language contexts, in which curricular subjects are taught through the medium of a second or \isi{foreign language}, can lead to very positive outcomes in the domains of receptive skills, vocabulary, morphology, speaking, creativity, and motivation (\citealt{Pérez-Cañado2012}). Regarding \isi{pragmatics}, research shows that integrated context and language classes provide opportunities for incidental \isi{pragmatic} learning \citep{Taguchi2015contextually}.


\section{Literature review}

The rationale and motivation for the present study are that, firstly, there are scarce data regarding the \isi{acquisition} of \isi{DMs} by second language (\isi{L2}) learners, despite their importance for \isi{successful communication}. Secondly, the ever growing importance of \ili{English}-medium instruction (\isi{EMI}) in Europe today has both social and political consequences. Thus, knowing more about language \isi{acquisition} in this setting can help inform higher education institutions across Europe regarding what types of \isi{linguistic} gains can be expected from participation in \isi{EMI} programs and what kind of language support is needed for students receiving their education through \isi{EMI}. The literature review consists of two parts: first, an overview of \isi{EMI} will be provided to contextualize the present study. In the second part, \isi{DMs} are identified and classified according to their functions, and studies examining their \isi{acquisition} are discussed.



\subsection{English-medium instruction}
\subsubsection{Policies}

While many factors have contributed to the rise of \isi{EMI} across Europe, the Bologna process was perhaps the most impactful (European Minister's of Education \citeyear{Bologna1999}). This large-scale policy change, which sought to encourage the mobility of students and faculty within Europe, had widespread effects on language policies across the European Union. In efforts to become more competitive and attractive to both faculty and students from other countries, universities began to offer degrees either partially or completely taught through languages other than the official language of the country, most notably \ili{English} \citep{Coleman2006,LlurdaEtAl2013,Pérez-Vidal2015forall}. In fact, the number of \isi{EMI} courses tripled from 1998 to 2008 in Europe (\citealt{WächterMaiworm2008}). The rapid implementation of \isi{EMI} programs continues to rise to this day reaching nearly 6\% of all programs offered in Europe (\citealt{SmitDafouz2012,WächterMaiworm2014}).


\subsubsection{Contextualization}

\isi{EMI} can be defined as a context in which \ili{English} is used as the language of instruction, in tertiary education, in non-\ili{English} speaking countries (\citealt{HellekjaerHellekjaer2015}). However, different regions and even individual universities have integrated \isi{EMI} into their specific context in unique ways, thus making \isi{EMI} somewhat of an umbrella term, for which specific realizations may differ across institutions. For instance, some regions have found it necessary to protect local languages, as was the case in the autonomous regions of Catalonia and the \ili{Basque} Country in Spain. When introducing \isi{EMI} programs in these two regions, the decision was made to implement trilingual policies with a view to protect and promote learning of regional languages (See \citealt{Pérez-Vidal2008}; and \citealt{DoizEtAl2014policy}). Similarly, Nordic countries question if there is perhaps an over-reliance on \ili{English} in academic contexts, and steps are being taken to protect national languages in research and education (\citealt{NordicCouncilofMinisters2006}). Thus, as demonstrated, program structure or intensity of \isi{EMI} differs according to each community’s language needs. Some may have full \isi{EMI} programs while others only a small percentage of \isi{EMI} courses. Institutions differ as well according to what type of language support is offered to students, faculty and administration (regarding both \ili{English} or national languages). Despite the differences in structure, when a course is offered through \isi{EMI}, there are also some constants, such as a strong focus on content and little attention or support offered to aid language learning. Although \isi{EMI} courses are now widespread, there is scarce research on how they are implemented in practice; only a handful of studies have been conducted, which reveal that lecturers do not focus on language, and that they may feel uncomfortable correcting errors as they are often non-native speakers of \ili{English} themselves \citep{Costa2012}. Lecturers are experts in their disciplinary fields and do not consider themselves language specialists; their aim, from their point of view, is therefore to educate students on their subject of expertise (\citealt{Airey2012,Unterberger2012}).


\subsubsection{Research on EMI}

Within the European Union, a body of research on \isi{EMI} from a second language \isi{acquisition} perspective has begun to emerge. Much of such research has taken a qualitative approach, investigating such topics as lecturers’ and students’ attitudes and beliefs towards \isi{EMI} \citep{Kling2013,KuteevaAirey2014}. These studies seek to inform policymakers, program creators, language teachers, and professional development departments about how \isi{EMI} is implemented in different institutions. Concerning \isi{content learning}, research reveals that students find courses harder and the workload heavier when taught through \ili{English} \citep{Tazl2011} and that \isi{EMI} is not perceived as equal to first language instruction in terms of content delivery \citep{Sert2008}. It has also been reported across a wide variety of contexts that students expect language gains when participating in \isi{EMI} programs \citep{PecorariEtAl2011,Gundermann2014,LuegLueg2015,MargićŽeželić2015}. However, as mentioned above, there is hardly any language support provided to students during \isi{EMI} degree programs, and language learning is not an explicit goal of such programs. Thus, investigating whether and how \isi{EMI} leads to gains in \isi{linguistic} competence is an area where more research is needed, and this chapter intends to offer a contribution in this direction.



\subsection{Discourse markers}


Discourse markers seem to play an important role both in first and second language \isi{acquisition}, since they are constantly used by native speakers (\isi{NSs}) and non-native speakers in interaction. As \citet{Yates2011} points out, \isi{DMs} help one interpret the speakers’ attitudes towards the content of their messages and they tend to carry socio-\isi{pragmatic} meaning. What some studies in \isi{SLA} have found is that foreign and second language learners tend to use a narrower variety of \isi{DMs} than \isi{NSs} do, and that they seem to be less aware of the multifunctional uses of \isi{DMs} \citep{Vanda2007,Yates2011}. However, even if \isi{DMs} seem to be key elements in interaction, defining and categorizing them is a complex issue, as the literature in the field has shown (see \citealt{Fischer2006} for review). First of all, different terms such as \isi{pragmatic} markers, discourse particles, discourse connectives, conversational markers, among others have been used to refer to these different \isi{linguistic} items which have specific cohesive functions and important interpretive roles in conversation. Secondly, the multifunctional nature of some \isi{DMs} has not been reflected in most of the categorizations presented so far, since most of these elements tend to have different functions depending on the context and situation where they are produced. Thirdly, one of the most problematic issues may be the grammaticalization of some \isi{DMs}, for some tend to overlap syntactically with subordinating conjunctions or coordinators, while some others may simply connect different parts of discourse. All these issues have contributed to categorizations that fail to completely and accurately describe what discourse markers really show in terms of structure and function \citep{Fischer2014}. 

Even if no clear definitions can be found in the literature, many studies have identified some common characteristics among \isi{DMs}. Most of them seem to show flexibility, that is, they are flexible in terms of their placement and use in discourse; additionally, they also encode speakers’ intentions and interpersonal meanings (\citealt{CarterMcCarthy2006}); they also carry little semantic meaning \citep{Schiffrin1987}, but at the same time are essential to the natural flow of speech, as well as to correct interpretation \citep{Neary-Sundquist2013}. Another aspect which has been reported in various studies is that hearers seem to rely on \isi{DMs} to interpret and follow discourse \citep{Blakemore1992,Aijmer1996}, so, as  \citep[156]{Leech1975} suggest, by using \isi{DMs} “in speech or in writing, you help people understand your message by signaling how one idea leads on from another. The words and phrases which have this connecting function are like ‘signposts’ on a journey”. Thus, these common characteristics are important elements for identifying discourse markers, which may have elusive \isi{referential} meanings on the surface, but play important roles on different planes of communication \citep{Schiffrin1987}.

In an attempt to describe how \isi{DMs} are used by non-native speakers, in the present study it was decided to follow existing categorizations in order to assess their adequacy for analyzing learners’ discourse. Therefore, following \citet{Maschler1994} and \citet{FungCarter2007}, the present study analyzes \isi{DMs} according to four functional categories: \textit{cognitive, structural, \isi{referential}}, and \textit{interpersonal}. Each category serves several related functions. \isi{DMs} in the \textit{cognitive} category are thought to provide information on the cognitive state of the speaker and instruct the hearer as to how to construct their mental representation of the ongoing discourse. \textit{Structural} \isi{DMs} serve metalinguistic textual functions on how the flow of discourse is to be segmented. \textit{Referential} \isi{DMs} mark relationships between the utterances before and after the DM; these relationships may be marked by conjunctions, and may be completely grammatically integrated while at the same time functioning pragmatically \citep{Fischer2014}, \isi{DMs} in this category seem to be more syntactically and textually bound than the other DM categories. The final category, \textit{interpersonal} \isi{DMs}, are thought to be used to mark affective and social functions of spoken grammar, and indicate how the speaker feels towards the discourse statements \citep{Andersen2001}. See \tabref{tab:ament:1} for a summary of the categories, functions and examples.

% \begin{table}
% \caption{\label{tab:ament:1} Categorization of pragmatic markers according to functions}
% 
% 
% \begin{tabularx}{\textwidth}{lQQ}
% \lsptoprule
% 
% \bfseries Category & \bfseries Functions & \bfseries Example Items\\
% \midrule
% Cognitive & {Denote thinking process}
% 
% {Reformulation / self-Correction}
% 
% {Elaboration / Hesitation}
% 
% Assessment of the listener’s knowledge about utterances & {\textit{Well, I think}}
% 
% 
% {\textit{In other words, I mean}}
% 
% 
% {\textit{It’s like / sort of, well}}
% 
% 
% \textit{Right?} \\
% 
% \tablevspace
% Structural & {Opening and closing of topics}
% 
% {Sequencing topic shifts}
% 
% {Summarizing options}
% 
% Continuation of or return to topics & {\textit{Ok, right, well, now,} }
% 
% {\textit{Anyway(s), so, then, next}}
% 
% {\textit{And, so yeah}}
% 
% \textit{Additionally, and so, and, plus} \\
% 
% \tablevspace
% Referential & {Cause / Contrast}
% 
% Consequence / Digression & {\textit{Because / But, although}}
% 
%  \textit{So /  Anyway}\\
% 
%  \tablevspace
% Interpersonal & {Mark shared knowledge}
% 
% {Indicate speaker attitudes} 
% 
% Show emotional response / interest and back channel & {\textit{You see, you know}}
% 
%  \textit{Yes, of course, really, I agree}
%  \textit{Great, sure, ok, yeah} \\
% \lspbottomrule
% \end{tabularx}
% \end{table}

\begin{table}
\caption{\label{tab:ament:1} Categorization of pragmatic markers according to functions}
 \begin{tabularx}{\textwidth}{Ql}
\lsptoprule 
\bfseries Functions & \bfseries Example items\\
\midrule
\textbf{Cognitive}\\
\midrule
{Denote thinking process} &{\textit{Well, I think}} \\ 
{Reformulation / self-Correction} & {\textit{In other words, I mean}}\\ 
{Elaboration / Hesitation}  &\textit{It’s like / sort of, well} \\ 
Assessment of the listener’s knowledge about utterances & \textit{Right?} \\
\tablevspace

\textbf{Structural}\\
\midrule 
{Opening and closing of topics}   &  {\textit{Ok, right, well, now,} } \\ 
{Sequencing topic shifts}           &   {\textit{Anyway(s), so, then, next}}  \\ 
{Summarizing options}               &  {\textit{And, so yeah}}  \\ 
Continuation of or return to topics &  \textit{Additionally, and so, and, plus}\\
\tablevspace


\textbf{Referential}\\
\midrule 
{Cause / Contrast} & {\textit{Because / But, although}}\\ 
Consequence / Digression &  \textit{So /  Anyway}\\

 \tablevspace
\textbf{Interpersonal}\\
\midrule 
{Mark shared knowledge} & {\textit{You see, you know}}\\
{Indicate speaker attitudes} &  \textit{Yes, of course, really, I agree}\\
Show emotional response / interest and back channel &  \textit{Great, sure, ok, yeah} \\
\lspbottomrule
\end{tabularx}
\end{table}



In the present study the use of \isi{DMs} and their relationship to \isi{pragmatic} competence is also explored, since, as \citep[1]{Müller2005} states “there is a general agreement that discourse markers contribute to the \isi{pragmatic} meaning of utterances and thus play an important role in the \isi{pragmatic} competence of the speaker”. Furthermore, \cite{SankoffEtAl1997} note that a learner’s use of \isi{DMs} may be a good indicator of the effect of \isi{L2} exposure on \isi{pragmatic} competence. The present study thus intends to investigate how \isi{DMs} may be acquired in \isi{EMI} contexts.



\subsection{The acquisition of pragmatic markers across contexts}
\subsubsection{Study abroad}

A sojourn abroad has proven to be a positive learning environment for the development of \isi{pragmatic} competence (\citealt{Barron2003,Schauer2006}). Cultural and \isi{linguistic} \isi{immersion} of this kind provides learners with increased opportunities to interact with \isi{NSs} of the \isi{target language}. Other benefits are that they are repeatedly exposed to daily routines, and have ample opportunities to practice a wide variety of communicative acts in many different social settings. These factors are believed to contribute to language learning. Research examining \isi{DMs} show positive results: for example, \citet{Liu2013,Liu2016} found that for \ili{Chinese} students living in the United States both the increased exposure and increased socialization had significant positive effects on the frequency and variety of \isi{DMs} produced. Similarly, \citet{Barron2003} measured the use of \isi{pragmatic} routines of 30 \ili{English}-speaking learners of \ili{German} through a written \isi{discourse completion task}. She found that the exposure to input in the \isi{target language} triggered important \isi{pragmatic} development and more target like use of \isi{pragmatic} routines. In a similar study on 128 international students who spent a study-abroad period in the United States, \cite{Sánchez-Hernández2016} (see also Chapter 10 of this volume) found parallel results, showing that there was a relationship between the degree of \isi{acculturation} and \isi{acquisition} of \isi{DMs}. These studies demonstrate how increased exposure, socialization and \isi{acculturation} through a study-abroad period have measurable effects on \isi{pragmatic} development.


\subsubsection{Classroom settings: formal instruction and integrated content and language classes}

While a classroom instructional setting does not offer the same variety of opportunities for learning or for practicing \isi{pragmatic} skills as studying abroad can, it does show benefits of its own. Both instructed as well as incidental learning of \isi{pragmatics} have been documented in previous studies (\citealt{NguyenEtAl2012,Bardovi-Harlig2015}). However, few studies report on the effects of \isi{explicit instruction} on the \isi{acquisition} of \isi{DMs}; rather, most studies take language samples from classroom learners and report on their usage of \isi{DMs} as learned incidentally. In the case of \citet{Bu2013}, oral data were gathered from interviews and recordings of \ili{English} classroom discussions. She found that the \ili{Chinese} learners in her study varied greatly regarding the types of \isi{DMs} used when compared to \isi{NSs}. She concluded that, while learners use many of the same \isi{DMs} as \ili{English} \isi{NSs}, they do not employ them with the same functions as \isi{NSs} do, and at times learners even give new and different functions to \isi{DMs} that \isi{NSs} never do (also found in \citealt{Müller2005}). A study on \ili{Chinese} learners of \ili{English} by \citet{Liu2013} also reported similar findings: Regarding the frequency of use of \isi{DMs} in interviews, some markers were used significantly differently when compared to \isi{NSs}. Those markers that learners used more frequently than the \isi{NSs} were \textit{just}, \textit{sort of/kind of, but}, \textit{well} and \textit{then}, compared to \textit{I think}, \textit{yeah/yes} and \textit{ah} which were used less frequently by the learners than by the \isi{NSs}. The author argues that the difference between learners’ and natives’ use of \isi{DMs} can be attributed to language transfer. 

Among the studies investigating a larger range of \isi{DMs}, \citet{Neary-Sundquist2014} studied the relationship between \isi{proficiency} level and \isi{pragmatic} marker use. She reported that DM use rose with \isi{proficiency} level, that lower \isi{proficiency} learners used \isi{DMs} much less frequently than \isi{NSs} did, and that advanced learners reach NS levels for the frequency of use. With respect to the variety of markers used, she found that low-level learners overuse certain expressions while advanced learners make use of a larger variety of \isi{DMs}. Another study comparing teenage learners in Hong Kong to a corpus of \ili{English} \isi{NSs} was conducted by \citet{FungCarter2007}. Through the analysis of interactions between students, they found that learners use \isi{referential} markers at high frequencies, while other categories are used more sparingly, and that \isi{NSs} use \isi{DMs} for a much wider variety of functions. The authors argue that the use of \isi{DMs} by the participants reflects the input they receive through their \isi{formal instruction} in \ili{English} courses. Regarding \isi{integrated content} and language settings, \citet{Nikula2008} studied adolescents’ communication in content courses taught in \ili{English}. She found this context to offer a wide variety of opportunities for \isi{pragmatic} learning and practice. From classroom observation she reported students using \isi{DMs} for a variety of \isi{pragmatic} functions, such as mitigating or softening their communication acts. This gives evidence that \isi{content learning} does allow learners to practice \isi{pragmatic} routines, such as \isi{DMs}. These studies show that although learners do not receive direct instruction on DM use, they can and do learn to use them implicitly, although more research is needed to know how the \isi{learning context} specifically affects the \isi{acquisition} of \isi{DMs}.


\section{The study}

The goal of the present research is to investigate the effect of the \isi{EMI} setting on the \isi{acquisition} and use of \isi{DMs}, in order to inspire future studies on similar larger populations and to provide empirical evidence as to what kinds of \isi{pragmatic} outcomes can be expected from the \isi{EMI} setting. 

The research questions addressed in the present study are as follows:

\begin{enumerate} 
\item Do \isi{EMI} and non-\isi{EMI} learners use \isi{DMs} at similar frequencies and distributions, according to the four functional categories of \isi{DMs}? 
\item Are there differences between the frequency and distribution of \isi{DMs} between \isi{EMI} and non-\isi{EMI} learners across tasks, viz. an \isi{oral discourse} completion task and a conversation task?
\end{enumerate}



\section{Methodology}


\subsection{Participants}


Sixteen second-year Economics undergraduate students from a university in Catalonia, Spain, were recruited to participate in this study. Results from a \isi{language background questionnaire} revealed that all participants were \ili{Spanish}/\ili{Catalan} bilinguals and that these languages were also the languages of their previous education. All participants reported \ili{English} as a third language. Participants reported having an \ili{English} certificate at a B2 level according to the common European framework of references for languages. 

Participants were divided into two groups: an \isi{immersion} group (henceforth, \isi{IM} group) (\textit{N} = 7, age = 19) and a control group (henceforth CON) (\textit{N} = 9, age = 19). The \isi{IM} group was enrolled in an International Business degree, which is taught completely through \ili{English}. Participants in the CON group were enrolled in either Economics or Business Administration at the same university but had only one of their courses taught through \ili{English} in the second year of study. Each degree program consists of 425 contact hours per academic year. For the \isi{IM} group, all 425 hours are delivered through the \ili{English} language, while the CON group had an exposure of 35 contact hours. Data were collected during the participants’ second year of study. \figref{fig:ament:1} illustrates the difference between groups in \ili{English} contact hours per academic year.

\begin{figure}
\caption{\label{fig:ament:1} Exposure to EMI} 

  \begin{tikzpicture}
    \begin{axis}[
	xlabel={},  
	ylabel={hours}, 
	axis lines*=left, 
        width  = .7\textwidth,
	height = .3\textheight,
    	nodes near coords, 
	xtick=data,
	x tick label style={},  
	ymin=0,
	ybar,
 	bar width=10mm,
    enlarge x limits=0.25,
	symbolic x coords={Year 1, Year 2},
	legend style={at={(1,0.4)},anchor=west}
	]
	\addplot+[ybar,lsRichGreen!80!black,fill=lsRichGreen] plot coordinates {
	    (Year 1, 425) (Year 2, 425)
	}; 
	\addplot+[ybar,lsMidBlue!80!black,fill=lsMidBlue] plot coordinates {
	    (Year 1, 0) (Year 2, 35)
	}; 
	\legend{\isi{IM} Group, Control group}
    \end{axis} 
  \end{tikzpicture} 
  
%  \begin{tikzpicture}
%      \begin{axis}[
%          width  = 11cm,
%          height = .23\textheight, 
%          symbolic y coords={good -- complete, no -- little},
%          xtick=data,
%          axis lines*=left,   
%          ybar,
%          bar width = 12pt,
%          nodes near coords,
%          nodes near coords align={horizontal},
%          axis on top,
%          xmajorgrids, tick align=inside,
%          major grid style={draw=white},
%          %enlargelimits=true,
%          enlarge y limits  = 0.4, 
%          legend style={at={(axis cs:11,good -- complete)},anchor=east}
%      ]    
%      \addplot+[lsMidBlue!80!black,fill=lsMidBlue] coordinates {(10,no -- little) (5,good -- complete)};
%      \addplot+[lsMidOrange!80!black,fill=lsMidOrange] coordinates {(7,no -- little) (8,good -- complete)};
%      \legend{nov, semi}
%      \end{axis}
%  \end{tikzpicture}

\end{figure}




\subsection{Data collection instruments}


Four instruments were used for data collection: two questionnaires (a \isi{language background questionnaire} and a \isi{proficiency} test) and two instruments for the elicitation of oral data (a conversation task, and an \isi{oral discourse} completion task).


\subsubsection{Language background and proficiency level questionnaires}

The \isi{language background questionnaire} was designed to gather information on the participants’ language background and \isi{learning history} to ensure homogeneity of the groups. The online Cambridge placement test was used to ensure a homogeneous group according to general \ili{English} \isi{proficiency}; any participant who did not score over a B2 level was not included in the sample.


\subsubsection{Conversation task}

In order to gather spoken data through interaction, participants were asked to engage in conversation with another participant. Participants were asked three questions that required them to reflect on and discuss their motivations and attitudes towards \ili{English} as a \isi{lingua franca}, as well as towards their \isi{EMI} courses (see appendix A).


\subsubsection{Oral discourse completion task}

A ten-item \isi{oral discourse} completion task was used to elicit \isi{DMs} (see appendix B). Discourse completion tasks as a research tool are supported by \cite{Usó-JuanMartínez-Flor2014}; \cite{ParvareshTavakoli2009}; \cite{KasperRose2002}, and \citet{Hinkel1997}, and they are particularly valuable for eliciting \isi{DMs} from \isi{L2} speakers \citep{Roever2009}. However, they have been strongly debated in the literature, mainly because participants are often asked to write what they would say in a certain situation and this is considered an inaccurate representation of what they would actually say in real-time communication \citep{Bardovi-Harlig2015}. 

In order to address these concerns, an audio and visual \isi{discourse completion task} was adopted. A video was created consisting of the researcher looking at the camera and recording the prompts for the ten discourse completion tasks, providing a pause of twenty-five seconds for the participants to respond before continuing to the next item. In this way, each item was orally contextualized and an interlocutor was provided to lower the cognitive load, thus enabling the participants to respond rapidly and as they would in an authentic interaction.



\subsection{Procedure}


Participants completed the web questionnaire and placement test via email before attending the testing session. Recording of data took place in sound-proof booths.  Each booth had a large window and was equipped with a microphone, headset and computer. The participants could see and hear the researcher outside of the booth and were given a series of instructions to set up Audacity, the program used to record their response. The \isi{oral discourse} completion task was administered first, by playing the video simultaneously on the participants’ computer screens. Twenty-five seconds were given to respond to each prompt. 

This was followed by the conversation task. For this task, participants were put into pairs in the booths, and recorded themselves. The researcher read each of the three questions out loud, the participants were told to include their opinions, personal experiences and anything else they felt they wanted to express in response to the statements. They had two minutes to discuss each question.



\subsection{Data analysis}


Audio files were transcribed into Codes for the Human Analysis of Transcripts (CHAT) using computerized language analysis (\isi{CLAN}) software \citep{MacWhinney2000}. The researcher identified and tagged each DM used in both the \isi{oral discourse} completion task and the conversation task and assigned it a code according to its functional category (cognitive, structural, \isi{referential}, or interpersonal). The transcriptions were then checked by another researcher. A single researcher coded the transcriptions twice to ensure consistency. A further 25\% of the transcriptions were coded by a second researcher; and when there was a discrepancy, an agreement was reached through discussing the item and together deciding on how it should be coded. After coding, the frequency of use of each type of DM was calculated using \isi{CLAN}.  Tables 2 to 5 provide extracts from the data, giving three examples of each function. 

\begin{table}
\caption{\label{tab:ament:2} Exemplification from the data, according to function: Cognitive DMs}

\begin{tabularx}{\textwidth}{lQ}
\lsptoprule
&\bfseries Examples from the data: Cognitive Function\\
\midrule

i &  \textit{Yeah I’ve tried it and it doesn’t fit me very well uhh I mean I would prefer another size or maybe another model that fits me better}\\

ii &  \textit{Please I’m not really well here ahh could you leave me alone for a minute please? It’s like I’m a bit sick and I don’t feel well I need some loneliness to recover myself please}\\

iii &  \textit{Umm well ahh I’m not sure it looks nice but I wouldn’t wear it}\\
%%
\lspbottomrule
\end{tabularx}
\end{table}


The markers were coded by taking into account the main function the DM was performing in the discourse, so that what may appear to be the same marker is, in fact, the marker performing different functions and therefore, would be coded accordingly. For example, the token of \textit{well} marks a cognitive function in example (\textit{iii}) in \tabref{tab:ament:2}, and was so coded because we stipulated that in this context (i.e. utterance initial and occurring between two hesitation markers such as \textit{umm}) that it signals a cognitive function (in this case, hesitation), and perhaps an effort to hold the floor while the speaker searches for a word or formulates their utterance in their mind. Looking at examples (\textit{i}) and (\textit{ii}) in \tabref{tab:ament:2}, \textit{I mean} functions to reformulate the message the speaker is conveying whereas \textit{it’s like} functions to signal an elaboration or exemplification of the previous utterance. 

\begin{table}
\caption{\label{tab:ament:3} Exemplification from the data, according to function: Structural DMs}

\begin{tabularx}{\textwidth}{lQ}
\lsptoprule
&\bfseries Examples from the data: Structural Function\\
\midrule

iv &  \textit{Yes and, umm, for example you can go to London and then you can go to U. S. and it’s totally different so you can also}
\\
v &  \textit{Participant 1: Umm ahh and the last ahh I would say that I see myself talking \ili{English} in well, I hope to be in in United States or or somewhere}\\

vi & \textit{Participant 2: hmm so I think that aah I I chose to to have lessons in \ili{English} because I wanted to improve my my level I wanted to to keep practising it}\\

vii &  \textit{stop bothering me you know you’re annoying me and my friends so I would really appreciate that you left right now}\\\\
%%
\lspbottomrule
\end{tabularx}
\end{table}


The structural markers were coded in the same manner, i.e. identifying the function of the marker in the discourse. For example, in example (\textit{iv}) in \tabref{tab:ament:3} the \isi{structural marker} \textit{and then} functions to show temporal sequence (going to one city and afterwards to another) and  also indicates an implied contrast between the two cities (the way \ili{English} is spoken contrasts greatly between the two cities).  In example (\textit{v}), \textit{and} functions to mark the summary of the speaker’s opinion on the matter being discussed and \textit{so} functions to mark the beginning of the speaker’s turn as well as a slight shift in the topic, a shift from participant one’s opinion to participant two’s opinion. In example (\textit{vi}), \textit{so} serves to summarize the speakers’ message. 

\begin{table}
\caption{\label{tab:ament:4} Exemplification from the data, according to function: Referential DMs}

\begin{tabularx}{\textwidth}{lQ}
\lsptoprule
&\bfseries Examples from the data: Referential Function\\
\midrule

vii &  \textit{I would suggest you to do it on the weekend because we don’t have so much homework from university}\\

viii &  \textit{I have tried on me but it doesn’t fit}\\

ix &   \textit{It don’t fit me because it’s so small so I have to change}\\
%%
\lspbottomrule
\end{tabularx}
\end{table}

The \isi{referential} marker \textit{because} in example (\textit{vii}) in \tabref{tab:ament:4} functions to introduce a reason or cause for suggesting the weekend for the party. Example (\textit{viii}) \textit{but} marks a contrast between the two parts of the utterance, and example (\textit{ix}) \textit{so} marks the causal or consequential relationship between the first part of the utterance and the second.  

\begin{table}
\caption{\label{tab:ament:5} Exemplification from the data according to function: Interpersonal DMs}

\begin{tabularx}{\textwidth}{lQ}
\lsptoprule
&\bfseries Examples from the data: Interpersonal Function\\
\midrule

x & \textit{Well I kind of you know we don’t have that much of a relationship with Laura and things have gone pretty badly lately.}
\\
xi & \textit{Yeah absolutely I love it but it’s a little bit small for my size}\\

xii & \textit{Oh yeah I love it but you know what it is too small}\\
%%
\lspbottomrule
\end{tabularx}
\end{table}


The use of the interpersonal marker \textit{you know} in example (\textit{x}) in \tabref{tab:ament:5} functions to align the speaker with their interlocutor and mark the shared knowledge that the speaker and the interlocutor have about the speaker and Laura not having a good relationship. Examples (\textit{xi}) \textit{yeah absolutely} and (\textit{xii}) \textit{oh yeah} are used to express the speaker’s attitudes and emotions towards what is being discussed.  Below, data from each task is provided. 

\ea\label{ex:ament:1}
 {Oral \isi{discourse completion task} data:} \\
 
1   \textbf{\textit{Well}} I'm not sure about it \textbf{\textit{you know}} Laura 
 
2   it’s a very chaotic girl \textbf{\textit{and}} she’s always 
 
3   making noise maybe it’s not such a good 
 
4   idea inviting Laura if you feel to do it 
 
5   ahh go ahead \textbf{\textit{but}} in my opinion she 
 
6   shouldn’t be invited \textbf{\textit{you know.}} 
\z

In \REF{ex:ament:1}, the speaker opens discourse and begins with the cognitive marker, \textit{well,} denoting mental processing. The participant begins to share her opinion and uses the interpersonal marker \textit{you know} to mark and confirm shared knowledge, with her interlocutor. She lets her interlocutor know that she is continuing to add information to the same topic using the \isi{structural marker} \textit{and}. Then in line 6, she uses the \isi{referential} marker \textit{but} to show contrast between what the speaker and hearer feel and restates her opinion, she finishes her turn by closing with the cognitive marker \textit{you know} as an attempt to align with her interlocutor as well as to assess the interlocutor’s reception of her message. 

\ea\label{ex:ament:2} 
 {Conversation data:} \\
 
1 \textbf{OK}, I start umm I’ve been learning \ili{English}  

2 all my life \textbf{\textit{and}}\textbf{ \textbf{I think}} that I I would be 

3 very competent and natural with \ili{English} 

4   speakers, native ones, \textbf{\textit{but}} \textbf{\textit{I think}} that 

5 I’ve always can improve. 
\z

In this example the participant first uses the structural DM \textit{OK} to mark the opening of discourse and begin her turn. Then another structural DM \textit{and} is used in line 2, in order to indicate a continuation of the topic and to add information. It is followed by the interpersonal DM \textit{I think} which gives an indication of the speaker’s personal opinion towards the statement immediately following the marker. Then in line 4, a \isi{referential} DM is used to contrast the information given after \textit{but} with the statement that precedes it. This is then immediately followed by an interpersonal DM \textit{I think} which expresses the speaker’s attitudes and beliefs towards the following statement.

Below are examples of data from the \isi{IM} group; P1, P2 etc. stand for Participant 1, 2 etc.  

\ea\label{ex:ament:3} 
P1:  We are colleagues in the same class.

P2:  Yes.

P1:  \textbf{So} we probably agree.

P2:  Yes the same.

P1:  \textbf{\textit{And}} how do you feel when you communicate with native \ili{English} speakers?

P2:  \textbf{\textit{Well}} I don’t feel comfortable.  
\z

In \REF{ex:ament:3}, participant 1 uses \textit{so} to summarize opinions with her statement ‘\textit{so,} we probably agree’. Then she uses \textit{and} as a \isi{structural marker} to signal a shift in the topic, from what the speaker feels to what participant two feels towards what is being discussed.  


\ea\label{ex:ament:4} 
P3:  Yes, it’s difficult to reach the level of \ili{English} that native speakers have, but I think that, umm it’s very important in, in your life to, to do so. \textbf{\textit{So}}. 
 
P4:  Yeah, \textbf{and} \textbf{well}, ahh, in, ahh, the future, I, I want to go to, for example, Londres (London), to find, to will find a homework, ahh because it’s a nice city. 
\z

In \REF{ex:ament:4}, participant 3 uses \textit{so} as a structural DM at the end of her utterance to sum up her opinion and mark the end of her turn, which participant 4 correctly interprets and uptakes with an appropriate response. She goes on to use \textit{and} as a topic shift marker and \textit{well} to mark the introduction of a new topic.


\section{Results}

This section first provides the results for \isi{research question} 1 by presenting the findings from the \isi{discourse completion task} and the conversation task together. Research question 2 is addressed in the second section by analyzing the two tasks separately. Before conducting inferential statistics, statistical assumptions were verified; for all but two of the variables, skewness and kurtosis values were out of normal distribution ranges. In addition, the sample size was small. It was thus decided to use non-parametric tests. Specifically, the Mann-Whitney test was carried out to detect any significant differences between the two groups of participants. Additionally, Cohen’s \textit{d} was calculated to determine effect sizes, using as a standardizer the pooled standard deviations of the two groups. The interpretation of the Cohen’s \textit{d} is as follows: \textit{d} values between 0 and .5 are considered small effect sizes, values between .5 and .8 are considered medium effect sizes, and over .8 are reflections of large effect sizes.



\subsection{Differences in frequency and variety of DM use according to the four categories across both tasks}


To begin descriptive statistics were first calculated. Most notably the results reveal that, despite being given equal amounts of time to complete the tasks, participants in the \isi{IM} group produced more words (\textit{M} = 847.86, \textit{SD} = 194), than the CON group (\textit{M} = 555.89, \textit{SD} = 166.59), which, according to the Mann-Whitney statistical test, proved to be significant, with a large effect size (\textit{U} = 8, \textit{p} = .013, \textit{d} = 1.614).  Furthermore, and probably as a consequence of this, the \isi{IM} group produced more \isi{DMs} (\textit{M} = 104.43, \textit{SD} = 24.61) compared to those in the CON group (\textit{M} = 72.22, \textit{SD} = 17.27) which also proved to be a significant difference, with a large effect size (\textit{U} = 8, \textit{p} = .013, \textit{d} = 1.515). However, with respect to the ratio of \isi{DMs} per 100 words, the CON group produced more than the \isi{IM} group: \isi{IM} (\textit{M} = 12.4, \textit{SD} = 1.47) versus CON (\textit{M} = 13.24, \textit{SD} = 1.42), although when tested for significance the result was not statistical (\textit{U} = 25, \textit{p} = .491). In order to assess the variety of \isi{DMs} used, Guiraud’s index was calculated, dividing the number of DM types by the square root of the number of DM tokens; the difference between groups was not statistically significant. Guiraud’s index is a corrected version of the standard type/token ratio (TTR), which is less sensitive to variations in text length \citep{Daller2010}. \tabref{tab:ament:6} reports descriptive and statistical results on the data from the two tasks together.  Due to the significant difference in number of words spoken, it was decided to calculate all further tests based on the percentage of \isi{DMs} produced with respect to the total number of words produced multiplied by one hundred.

\begin{table} 
 
\caption{\label{tab:ament:6} Descriptive and statistical data for both tasks IM and CON groups}
\begin{tabularx}{\textwidth}{l Q@{~}p{18mm}p{18mm}Q}
\lsptoprule
\bfseries Group (N) & \bfseries Words\newline spoken & \bfseries {DMs}\newline spoken & \bfseries {DMs} per\newline 100 words & \bfseries Mean DM’ 
\mbox{Guiraud Index}\\
\midrule
{{{IM} (7)}} & {M = 847.86} & M = 104.43  & M = 12.4  & {M = 1.73}\\
             & SD = 194     &  SD = 24.61 & SD = 1.47 & SD = .13\\

% \tablevspace
{CON (9)} & M = 555.89  & M = 72.22   & M =13.24  & {M = 1.76}\\
	  & SD = 166.59 &  SD = 17.27 & SD = 1.42 & SD = .19\\

% \tablevspace
\mbox{Mann-Whitney} test& {U = 8}  & {U = 8}  & {U = 25} & {U = 18.5} \\
		        & p = .013 & p = .013 & p = .491 & p = .19\\

% \tablevspace
{Cohen’s d} & d = 1.614 & d = 1.515 & d = - 0.58 & d = -0.20\\
\lspbottomrule
\end{tabularx}
\end{table}


In order to respond to \isi{research question} 1 – \textit{Do \isi{EMI} and non-\isi{EMI} learners use \isi{DMs} at similar frequencies and distributions, according to the four functional categories of \isi{DMs}?} – Further analyses with respect to the four functional categories were carried out. \tabref{tab:ament:7} shows the mean ratios of \isi{DMs} produced per participant according to each category across both tasks as well as the mean percentage of occurrence of each category of DM with respect to the total \isi{DMs} produced. Regarding this distribution, when both tasks were analyzed together, the \isi{IM} group produced a higher proportion of cognitive (\isi{IM} = 12.72\%, CON = 11.85\%), structural (\isi{IM} = 24.49\%, CON = 18.46\%) and interpersonal markers (\isi{IM} = 37.62\%, CON = 36.92\%), while the CON group tended to produce a higher rate of \isi{referential} markers (\isi{IM} = 23.94\%, CON = 30.77\%). The CON group also produced more DM tokens and types per 100 words, which is reflected in the slightly larger value of the Guiraud Index. This may indicate that the use of \isi{DMs} was both more frequent and more varied than compared to the \isi{IM} group.

\begin{table}[t]
\caption{\label{tab:ament:7} DMs used according to DM category IM and CON group both tasks} 
 
\begin{tabularx}{\textwidth}{Q rrr@{\quad}rrr}
\lsptoprule
& \multicolumn{4}{c}{\bfseries \isi{IM} Group} & \multicolumn{2}{c}{\bfseries CON Group}\\
\bfseries DM Category & \bfseries {Mean} & \bfseries {SD} & \bfseries {\% of all DMs} & \bfseries {Mean} & \bfseries {SD} & \bfseries {\% of all DMs}\\
\midrule
{Cognitive} & 1.58 & .46 & 12.72 & 1.56 & .56 & 11.85\\
{Structural} & 3.02 & .96 & 24.49 & 2.40 & .67 & 18.46\\
{Referential} & 2.89 & 1.01 & 23.94 & 4.02 & 1.25 & 30.77\\
{Interpersonal} & 4.74 & 1.13 & 37.62 & 5.01 & 1.57 & 36.92\\
{DM frequency \mbox{(tokens per 100w)}}   & 12.41 & 1.47 & n/a & 13.24 & 1.42 & n/a\\
{DM variety \mbox{(types per 100w)}}    & 3.07 & .74 & n/a & 3.97 & .74 & n/a\\
\lspbottomrule
\end{tabularx}
\end{table}


  

A Mann-Whitney test was carried out in order to detect any significant differences between the groups regarding these values per 100 words. Results show there was a significant difference in the production of \isi{referential} markers, with a large effect size (\textit{U} = 7, \textit{p} = .010, \textit{d} = 1.097). Specifically, the CON group (\textit{M} = 4.33, \textit{SD} = 1.25) produced more \isi{referential} \isi{DMs} than the \isi{IM} group (\textit{M} = 3.32, \textit{SD} = 1.01). Results for the remaining variables measured were not significant. The probability values for the differences and effect sizes are reported in \tabref{tab:ament:8}.  

\begin{table} 
\caption{\label{tab:ament:8} Comparison of IM and CON groups} 
% \small
\begin{tabularx}{\textwidth}{Xrrr}
\lsptoprule
\bfseries Category of DM & \bfseries {Mann-Whitney Value} & \bfseries {p-value}  & \bfseries {Cohen’s d}   \\
\midrule
{Cognitive DMs}  &  31 & .958 & {} .039  \\
{Structural DMs}  &  20 &  .223 & {{} .749}\\
{Referential DMs}  & { 14} &  .064 & {{ -1.536}}\\
{Interpersonal DMs}  &  26 &  .560 & {{} .197}\\
\mbox{{DM frequency (tokens per 100w)}}    &  25 &  .491 & {{} .581}\\
\mbox{{DM variety (types per 100w)}}    &  13 &  .055 & {{ -}1.211}\\
{Guiraud’s Index} & {40} & .40 & {{} -0.195}\\
\lspbottomrule
\end{tabularx}
\end{table}


To summarize the results from \isi{research question} 1, it was found that \isi{IM} students spoke significantly more, and produced significantly more \isi{DMs} in their texts, in absolute terms. However, looking at standardized values per 100 words, there were no significant differences detected between the groups. Regarding the distribution of the different categories of \isi{DMs}, the CON group was found to produce a significantly higher ratio of \isi{referential} \isi{DMs} than the \isi{IM} group.



\subsection{Differences in frequency and variety of DM use in each task separately}


Separate analyses were run for each task in order to address \isi{research question} 2 - \textit{Are there any differences between groups depending on the task, according to the four categories?-} Regarding the \isi{discourse completion task}, descriptive statistics were calculated (see \tabref{tab:ament:9}) and a Mann-Whitney test was then carried out to detect statistical significance (see \tabref{tab:ament:10}). As in the previous section, all values discussed here are based on ratios per 100 words, given the significant differences in text length between the two groups. 





\begin{table}[t]
\caption{\label{tab:ament:9} Descriptive statistics for the oral discourse completion task}
\begin{tabularx}{\textwidth}{Q rr@{~}r rr@{~}r}
\lsptoprule
\bfseries DM Category & \bfseries {Mean} & \bfseries {SD} & \bfseries {\% of all DMs} & \bfseries {Mean} & \bfseries {SD} & \bfseries {\% of all DMs}\\
\midrule 
{Cognitive} & 1.06 & .41 & 10.88 & 1.65 & 1.17 & 11.87\\
{Structural} & 2.04 & 1.00 & 20 & 1.94 & 1.09 & 15.73\\
{Referential} & 2.38 & 1.16 & 24.56 & 3.78 & .91 & 28.78\\
{Interpersonal} & 4.19 & 1.12 & 41.40 & 5.38 & 2.09 & 39.76\\
{Mean~words} & 399.57 & 84.72 & n/a & 287.33 & 100.34 & n/a\\
\mbox{DM~frequency} \mbox{(tokens per 100w)}    & 10.01 & 1.78 & n/a & 13.19 & 2.30 & n/a\\
\mbox{DM~variety} \mbox{(types per 100w)}    & 2.79 & .49 & n/a & 3.49 & 1.48 & n/a\\
\mbox{Guiraud’s Index} & 1.74 & .20 & n/a & 1.54 & .34 & n/a\\
\lspbottomrule
\end{tabularx}
\end{table}

\begin{table}[t]
\caption{\label{tab:ament:10} Comparison of groups discourse completion task }
\begin{tabularx}{\textwidth}{Qrrr}
\lsptoprule
\bfseries Category of DM & \bfseries {Mann-Whitney value} & \bfseries {p-value}  & \bfseries {Cohen’s d} \\
\midrule 
{Cognitive} &  21 &  .266 & { -0.742}\\
{Structural} &  31 &  .958 & {} .095\\
{Referential} &  12 &  .039 & {} 1.34\\
{Interpersonal} &  20 &  .223 & {} .737\\
{Mean Words}  &  12 &  .039 & {} 1.209\\
\mbox{{DM frequency (tokens per 100w)}}   &  7 &  .010 & {} 1.546\\
\mbox{{DM Variety  (types per 100w)}   }&  23 &  .401 & {-0.722} \\
\mbox{Guiraud’s Index} &  19 &  .186 & {} .752\\
\lspbottomrule
\end{tabularx}
\end{table}

The \isi{IM} group (\textit{M} = 399.57, \textit{SD} = 84.72) produced significantly more words than the CON group, with a large effect size (\textit{M} = 287.33, \textit{SD} = 100.34) (\textit{U} = 12, \textit{p} = .039, \textit{d} = 1.209). According to the distribution of \isi{DMs}, results show tendencies for the \isi{IM} group to produce a higher rate of structural (\isi{IM} = 20.00\%, CON = 15.73\%) and interpersonal markers (\isi{IM} = 41.40\%, CON = 39.76\%) than the CON group, while the CON group appears to produce higher rates of cognitive (\isi{IM} = 10.88\%, CON = 11.87\%) and \isi{referential} markers (\isi{IM} = 24.56\%, CON = 28.78\%) than the \isi{IM} group. The only significant difference between the groups was detected in the \isi{referential} marker category, with a large effect size (\textit{U} = 12, \textit{p} = .039, \textit{d} = 1.34). 

However, despite speaking more, the results show the \isi{IM} group (\textit{M} = 10.01, \textit{SD} = 1.78) produced significantly fewer \isi{DMs} per 100 words than the CON group (\textit{M} = 13.19, \textit{SD} = 2.30), with a large effect size (\textit{U} = 7.00, \textit{p} = .010, \textit{d} = 1.546). Furthermore, the CON group (\textit{M} = 3.49, \textit{SD} = 1.48) was found to produce a wider variety of DM types than the \isi{IM} group (\textit{M} = 2.79, \textit{SD} = .49), although the result was not significant. Variety of types per 100 words is a measure that can be partially affected by text length (for example, longer texts will tend to have more repetitions of the same types). The Guiraud Index, which introduces a partial correction for these effects, is in fact slightly higher in the \isi{IM} group (\textit{M =} 1.74, \textit{SD} = .20) than the CON group (\textit{M} = 1.54, \textit{SD} = .34), although this difference was not significant either. 

In sum, significant differences were that the \isi{IM} group spoke more than the CON group and that the CON group produced a higher frequency of \isi{DMs} per 100 words, as well as a significantly higher proportion of \isi{referential} \isi{DMs} than the \isi{IM} group.


Turning to the conversation task, descriptive statistics were calculated first (see \tabref{tab:ament:11}), and secondly the data were analyzed statistically using the Mann-Whitney test (see \tabref{tab:ament:12}). The descriptive statistics show that, during the conversation task, the \isi{IM} group produced more words (\textit{M} = 448.28, \textit{SD} = 143.40), than the CON group (\textit{M} = 268.56, \textit{SD} = 84.14) a difference that proved to be statistically significant, with a large effect size (\textit{U} = 9, \textit{p} = .017, \textit{d} = 1.529). The \isi{IM} group also showed a higher frequency of DM production overall (\textit{M} = 14.90, \textit{SD} = 3.12) compared to the CON group (\textit{M} = 13.08, \textit{SD} = 1.54), however, this difference failed to prove significant. The CON group produced a higher variety of \isi{DMs} (\textit{M} = 4.44, \textit{SD} = 77) compared to the \isi{IM} group (\textit{M} = 3.35, \textit{SD} = 1.42), and the difference was significant, with a large effect size (\textit{U} = 10, \textit{p} = .023, \textit{d} = .954). 

\begin{table}
\caption{\label{tab:ament:11} Descriptive statistics for the conversation task} 
\begin{tabularx}{\textwidth}{Q rr@{~}r rr@{~}r}
\lsptoprule
& \multicolumn{3}{c}{\bfseries \isi{IM} Group} & \multicolumn{3}{c}{\bfseries CON Group} \\
\bfseries DM Category & \bfseries Mean & \bfseries SD & \bfseries \% of all \isi{DMs} & \bfseries Mean & \bfseries SD & \bfseries \% of all \isi{DMs}\\
\midrule 
{Cognitive} & 1.95 & .88 & 13.90 & 1.49 & 1.24 & 11.82\\
{Structural} & 4.11 & 1.42 & 27.35 & 2.75 & 1.02 & 21.41\\
{Referential} & 3.31 & 1.01 & 23.56 & 4.33 & 1.25 & 32.91\\
{Interpersonal} & 5.52 & 2.53 & 35.20 & 4.51 & 1.58 & 33.87\\
{Mean words} & 448.28 & 143.40 & n/a & 268.56 & 84.14 & n/a\\
\mbox{DM frequency} \mbox{(tokens per 100w)}   & 14.90 & 3.12 & n/a & 13.08 & 1.54 & n/a\\
\mbox{DM Variety} \mbox{(types per 100w)}  & 3.35 & 1.42 & n/a & 4.44 & .77 & n/a\\
\mbox{Guiraud’s Index} & 1.72 & .30 &  & 1.98 & .37 & \\
\lspbottomrule
\end{tabularx}
\end{table}

\newpage 
Concerning the categories of \isi{DMs}, a statistically significant difference was detected in the use of structural \isi{DMs}, with a large effect size (\textit{U} = 13, \textit{p} = .050 \textit{d} = 1.100). Specifically, the \isi{IM} group (\textit{M} = 4.11, \textit{SD} = 1.42) produced more structural \isi{DMs} than the CON group (\textit{M} = 2.75, \textit{SD} = 1.02). Furthermore, the \isi{IM} group tended to produce more cognitive and interpersonal \isi{DMs}, and the CON group more \isi{referential} \isi{DMs}, although none of these differences were significant. 

To summarize results from the conversation task, it was found that the \isi{IM} group produced significantly more words and more structural \isi{DMs} than the CON group and that the CON group produced a significantly higher variety of \isi{DMs} than the \isi{IM} group. 


\begin{table}[t]
\caption{\label{tab:ament:12} Comparison of Groups Conversation Task (ratios per 100 words)}
\begin{tabularx}{\textwidth}{Qrrrr}
\lsptoprule
\bfseries Category of DM & \bfseries {Mann-Whitney value} & \bfseries {p-value}  & \bfseries {Cohen’s d} \\
\midrule 
{Cognitive} &  24 &  .427 & {} .428\\
{Structural} &  13 &  .050 & {} 1.100\\
{Referential} &  14 &  .064 & { -0}.898\\
{Interpersonal} &  24 & .427 & {} .479\\
{Mean Words}  &  11 & .034 & {} 1.529\\
\mbox{DM frequency (tokens per 100w)}   &  21 & .226 & {} .740\\
\mbox{DM Variety (types per 100w)}  &  10 & .023 & {} .954\\
\mbox{Guiraud’s Index} & {16} & {.10} & {}.772\\
\lspbottomrule
\end{tabularx}
\end{table}


\section{Discussion}

This study did not find many statistically significant differences between the two groups, both because of the limited sample size and also because the two groups were rather similar with respect to several dimensions. However, the significant differences that were found offer some interesting points for discussion.

Regarding the first \isi{research question}, findings show that students in an \isi{EMI} program produced longer responses and dialogues. When calculating absolute scores, \isi{EMI} students produced significantly longer stretches of speech and a significantly higher number of \isi{DMs}. Both of these findings can be considered signs of increased oral \isi{fluency} (\citealt{SegalowitzFreed2004}). However, when text length was controlled for via calculation of standardized values per 100 words, the differences were not sustained. While this points out that the two groups produce similar frequencies of \isi{DMs} in proportion to the total length of text produced, it still evidences an increased oral \isi{fluency} among the \isi{EMI} students.

The second finding was that the non-\isi{EMI} group had a very high frequency of use of \isi{referential} markers. Previous research has suggested that this category might be easier or could be the first category of \isi{DMs} to be acquired \citep{Liu2016}. This is due to the main functions of \isi{referential} \isi{DMs}, namely to show cause and contrast, consequence and comparison. These markers are the type of DM most often addressed in the \isi{foreign language classroom} due to their close relationship with syntax  as well as their strong prevalence in written language (\citealt{FungCarter2007}). This contrasts with the other categories which appear more frequently or even exclusively as oral markers and have fewer text-dependent functions \citep{Andersen2001}. While the \isi{EMI} students did integrate \isi{referential} markers into their speech, they did not use them quite as frequently as the non-\isi{EMI} group did; on the contrary, they had a slightly more even distribution of use of \isi{DMs} over the four categories, which may be an indication of the \isi{EMI} group employing a more appropriate distribution of \isi{DMs} across functions. It might be the case that \isi{EMI} students were able to select other more appropriate markers while the CON group seemed to rely more on \isi{referential} markers. These findings echo those reported in \citet{FungCarter2007}, who found that \isi{L2} learners relied on \isi{referential} \isi{DMs} more than on the other DM categories. 

Turning to the interpretation of the results in terms of  the second \isi{research question}, it was found that the non-\isi{EMI} group produced a higher ratio of \isi{DMs} to words during the \isi{discourse completion task}. A possible explanation for this result may be that, due to the strict time limit during the \isi{discourse completion task}, there may have been some \isi{cognitive competition} as described by \citet{Skehan1998}, where some features are attended to at the expense of others. For example, in this case, providing a response within the time given may have been a difficult task for the non-\isi{EMI} participants and, as a consequence, little attention might have been paid to how the message was delivered; in other words, they may have been more likely to repeat the same markers and utilize the same sentence structures to organize their discourse and convey their ideas to their interlocutors. This may be due to being unsure of how to continue a natural flow of conversation during the task. This interpretation would account for the difference in production of \isi{DMs} between the groups. This effect of \isi{cognitive competition} could be more prevalent in the non-\isi{EMI} group, as they may speak \ili{English} less often and might be less used to spontaneously using \ili{English}, whereas the \isi{EMI} students are accustomed to using \ili{English} daily and thus might able to use \isi{DMs} slightly more selectively.

Additionally, the \isi{EMI} participants were found to produce significantly longer texts, which as mentioned above can be interpreted as a sign of \isi{fluency}, since they were able to produce longer responses than the non-\isi{EMI} group was in the same amount of time. We suggest this could be due to the constant and frequent exposure to \isi{EMI} classes. However, in future research one might compare the results to NS data to confirm if the ratio of \isi{DMs} produced by the groups is similar or different from NS usage. 

  
Regarding the significant difference between the use of \isi{referential} \isi{DMs} as measured on the \isi{discourse completion task}, this trend was also found when analyzing the two tasks together and has been discussed above. It seems that the \isi{referential} category is more closely linked to grammar and what is taught in \isi{L2} classrooms. The functions of \isi{referential} \isi{DMs} appear to be the most transparent in their meaning and use, and thus, may be slightly easier to incorporate into the one’s speech than the other DM categories \citep{Liu2016}.

Turning to the conversation task, in addition to producing significantly longer responses, which has already been discussed, \isi{EMI} students were found to use significantly more structural markers than the non-\isi{EMI} group during the conversation task. This could be a reflection of a slightly higher or more sensitive \isi{pragmatic} competence in their ability to signpost discourse while engaged in conversation, as was found in \citet{Wei2011}, whose advanced learners were reported to use more structural markers to highlight information. Furthermore, the use of structural markers could be a sign of increased \isi{linguistic} complexity. This finding aligns with those from \citet{Neary-Sundquist2014}, who reported that higher \isi{proficiency} learners used \isi{DMs} to support and enable their \isi{fluency}, and that as \isi{proficiency} increases, learners can allocate more attention not only on delivering their message but on how they would like their message to be received. However, in the present study, our participants had the same \isi{proficiency} and they only differed in terms of amount of exposure to the \isi{target language}. This leads us to suggest that the number of hours of exposure available through \isi{immersion} programs (as was the case for the \isi{IM} group in this study), may provide learners with more opportunities for communication and thus make them more aware of how they express themselves while speaking \ili{English}.

\largerpage
It was also found that the non-\isi{EMI} group produced a wider variety of \isi{DMs} overall compared to the \isi{EMI} group during the conversation task. It seems that text length could be playing a strong role here. The \isi{EMI} participants produced significantly longer responses on all tasks, and it is therefore much more likely that in a long text the same markers are used more than once. This interpretation is further supported by the non-significant findings of differences in variety found according to Guiraud’s Index. When text length was controlled for, the significant difference between the groups was not sustained.

As mentioned in the literature review, the type of input in \isi{EMI} is mainly via lectures (\citealt{HellekjaerHellekjaer2015}), a context where academic language with a formal tone is primarily used. Lecturers must cue their interlocutors as to when they are opening a topic, changing, returning to, or continuing a topic, as well as mark progression while explaining processes. These functions are carried out by structural markers \citep{Andersen2001}, thus, making them one of the most salient categories of \isi{DMs} that \isi{EMI} students are exposed to. This may be why \isi{EMI} students integrate more structural markers into their speech than the non-\isi{EMI} students. 

Despite the explanations provided as possible reasons for the differences according to task, the results do not seem to point towards a clear relationship between task and DM use, as was also found by \citet{Neary-Sundquist2013}. This clearly points to the need for more research in this area.


\section{Conclusion}
\largerpage
This preliminary study seems to provide evidence that the context of learning can make some difference in the learning of \isi{pragmatics}. \isi{EMI} students were found to produce significantly longer responses than the non-\isi{EMI} group, including more words and more \isi{DMs} in absolute terms, which is a sign of increased oral \isi{fluency}. Furthermore, \isi{EMI} students produced more structural \isi{DMs}, which showed an effort on the behalf of these participants to produce more complex language and to signpost discourse clearly. The \isi{EMI} students also had a more even distribution of use of \isi{DMs} across categories. This could be a reflection of development in \isi{pragmatic} competence: It seems as though the increased amount of time spent in \isi{EMI} classrooms may lead learners to attend more to how they want their messages to be interpreted by their interlocutors. This pattern of use also reflects the type of input they receive, namely, academic lectures. Non-\isi{EMI} students, on the other hand, produced more \isi{referential} \isi{DMs}, which seems to be the first category learned due to their transparent meanings, attention given to them in language classrooms as well as their prevalence in writing and formal speech (\citealt{FungCarter2007,Neary-Sundquist2014,Liu2016}). 

This study aimed to shed some light on the incidental \isi{acquisition} of DM in the \isi{EMI} classroom and we have identified some trends. However, the study was conducted on a small number of participants and the findings should be taken as preliminary. It is, therefore, important to carry out more studies in this context with more participants to confirm the trends found here. 

\section*{Acknowledgments}

The authors extend their gratitude to a number of researchers who offered their valuable insights comments and support during this project, Eloi Puig Mayenco, Andrew Lee, and Dr. Roy Lyster, as well as to the anonymous reviewers and editors of this monograph for their suggestions.

\section*{Funding}

This study was supported by the \ili{Spanish} Ministry of Science and Innovation (grant FFI2013-48640-C2-1-P). 

\sloppy
\section*{Appendix A: Conversation task}

\begin{enumerate}
\item Do you imagine yourself being a completely competent and natural speaker of \ili{English} in the future? How do you feel when communicating with native speakers of \ili{English}? What place do you see \ili{English} having in your future?
\item Why do you believe courses are taught in \ili{English} in your University? Why did you enroll in a degree program that is taught in \ili{English}? How do you feel about being taught in \ili{English} by non-native speakers of \ili{English}? 
\item Do you enjoy communicating in \ili{English} with other Non-Native \ili{English} speakers? Can you share any of your experiences using \ili{English} as an international language? 
\end{enumerate}

\section*{Appendix B: Oral discourse completion task}

 \begin{enumerate}
\item \textbf{Contextualization}: Your best friend is inviting you to her birthday party. You will definitely be able to make it whenever she suggests because she is such a good friend. (Suggestion non-face-threatening)

\item \textbf{Researcher} \textbf{on} \textbf{video} \textbf{speaking} \textbf{directly} \textbf{to} \textbf{participant}: Hi, so I have just about everything for the party planned, which day do you think I should have it? 

\item  \textbf{Contextualization}: Your friend wants to invite Laura to the birthday party, a girl that your friend knows you don’t get along with. Try to convince your friend to not invite Laura.  (Suggestion face-threatening) 

\item \textbf{Researcher} \textbf{on} \textbf{video} \textbf{speaking} \textbf{directly} \textbf{to} \textbf{participant:} Oh yes, and by the way, I ran into Laura the other day, we went out for coffee. I know you’re not crazy about her, but I invited her to my birthday party. That will be ok, won’t it? 

\item \textbf{Contextualization}: Your friend is telling you all about her birthday plans; tell her what you think of them. (Opinion, non-face-threatening)

\item \textbf{Researcher} \textbf{on} \textbf{video} \textbf{speaking} \textbf{directly} \textbf{to} \textbf{participant:} As you know it’s my birthday coming up next week, and I have a few ideas about what I’d like to do. I thought about inviting everyone for dinner at my house, maybe everyone could bring a dish, then, afterwards we could go out and celebrate in this bar I know where you can drink and dance. 

\item \textbf{Contextualization}: You are shopping with a friend, they are trying on a hat that you think is very old-fashioned looking, and the colour (red) is terrible. You don’t like it at all. (opinion face-threatening) 

\item \textbf{Researcher} \textbf{on} \textbf{video} \textbf{speaking} \textbf{directly} \textbf{to} \textbf{participant:} Oh, I love just love hats, all kinds really. This red one is quite nice. What do you think, does it suit me? 

\item \textbf{Contextualization}: Your friends gave you a sweater as a gift. You don’t really like it and you want to return it. You need to ask your friend for the receipt so you can exchange it. (request, face threatening)

\item \textbf{Researcher} \textbf{on} \textbf{video} \textbf{speaking} \textbf{directly} \textbf{to} \textbf{participant:} So, have you had time to try on the sweater? Does it fit? We all hope you like it.

\item \textbf{Contextualization}: You are meeting your friend for a coffee and just missed the train; you’ll now be a few minutes late. (apology non-face threatening) 

\item \textbf{Researcher} \textbf{on} \textbf{video} \textbf{speaking} \textbf{directly} \textbf{to} \textbf{participant:} Hi, I am here waiting. Where are you? 

\item \textbf{Contextualization}:  Your friend’s party started at 10. It is now 11 and you will not be able to go at all. You know she is going to be very disappointed. You call her and tell her. (apology, face threatening) 

\item \textbf{Researcher} \textbf{on} \textbf{video} \textbf{speaking} \textbf{directly} \textbf{to} \textbf{participant:} Hi, where are you? Are you on your way? 

\item \textbf{Contextualization}: Your friend has just picked you up in their car, and has all the windows down. You are cold and need to ask them to turn on the heat or roll up the windows. 

\item \textbf{Researcher} \textbf{on} \textbf{video} \textbf{speaking} \textbf{directly} \textbf{to} \textbf{participant:} Nothing, researcher provides interlocutor only. 

\item \textbf{Contextualization}: Your friend gets to the party and really looks great. You can tell that they cut their hair and have bought new clothes. You want to tell them how good they look. (Compliment, non-face threatening) 

\item \textbf{Researcher} \textbf{on} \textbf{video} \textbf{speaking} \textbf{directly} \textbf{to} \textbf{participant:} Nothing, researcher provides interlocutor only. 

\item \textbf{Contextualization}: You have been talking to this person at the party for a while and they are really starting to bother you. They keep making fun of your friends and you find it insulting, you find them offensive. You have tried to walk away, but they keep cornering you. You will have to tell them to leave you alone. (aggressive situation) 

\item \textbf{Researcher} \textbf{on} \textbf{video} \textbf{speaking} \textbf{directly} \textbf{to} \textbf{participant:} Nothing, researcher provides interlocutor only.

\end{enumerate}
 

\sloppy
\printbibliography[heading=subbibliography,notkeyword=this] 
\end{document}